
\documentclass[a5paper,12pt,landscape]{article}

\usepackage[a5paper,margin=8mm,bmargin=16mm]{geometry}
\usepackage[german,main=french]{babel}

\usepackage{fontspec}
\usepackage{microtype}

\setmainfont{EBGaramond}
[
  Extension      = .otf,
  UprightFont    = *-Regular,
  ItalicFont     = *-Italic,
  BoldFont       = *-Bold,
  BoldItalicFont = *-BoldItalic,
  Numbers        = Lowercase,
  Ligatures      = Discretionary,
  Style          = Swash
]
% https://texnique.fr/osqa/questions/8423/lualatex-et-ligatures

\usepackage{paracol}

\newcommand\chunks[2]{%
\begin{leftcolumn*}\begin{otherlanguage}{german}%
{#1}%
\end{otherlanguage}\end{leftcolumn*}%
\begin{rightcolumn}\begin{otherlanguage}{french}%
{#2}%
\end{otherlanguage}\end{rightcolumn}%
}
% https://tex.stackexchange.com/a/505562/295527

\begin{document}

\thispagestyle{empty}
\setlength{\columnsep}{8mm}
\setlength{\columnseprule}{0.4pt}

% https://korpora.org/kant/aa09/025.html

% Zur Philosophie nach dem Schulbegriffe gehören zwei Stücke:
% Was aber Philosophie nach dem Weltbegriffe (in sensu cosmico) betrifft:
% Denn Philosophie in der letztern Bedeutung ist

% Dans la philosophie selon sa notion scolastique, il faut faire deux parties :
% Mais s'agissant de la philosophie selon son sens cosmique, on peut aussi l'appeler
% Car la philosophie, en ce dernier sens, est

\begin{paracol}{2}
  %\chunks{\begin{center}
  %        \textbf{Das Feld der Philosophie}
  %        \end{center}}
  %       {\begin{center}
  %        \textbf{Le domaine de la philosophie}
  %        \end{center}}
  \chunks{\subsection*{Das Feld der Philosophie.}}
         {\subsection*{Le domaine de la philosophie.}}
  \chunks{Das Feld der Philosophie in dieser weltbürgerlichen Bedeutung läßt sich auf folgende Fragen bringen: 	  	  	 

\begin{enumerate}
  \item Was kann ich wissen?
  \item Was soll ich thun?
  \item Was darf ich hoffen?
  \item Was ist der Mensch?
\end{enumerate}

Die erste Frage beantwortet die \emph{Metaphysik}, die zweite die \emph{Moral}, die dritte die \emph{Religion} und die vierte die \emph{Anthropologie}. Im Grunde könnte man aber alles dieses zur Anthropologie rechnen, weil sich die drei ersten Fragen auf die letzte beziehen.}
         {Le domaine de la philosophie en ce sens cosmopolite se ramène aux questions suivantes :

\begin{enumerate}
  \item Que puis-je savoir ?
  \item Que dois-je faire ?
  \item Que m'est-il permis d'espérer ?
  \item Qu'est-ce que l'homme ?
\end{enumerate}

À la première question répond la \emph{métaphysique}, à la seconde la \emph{morale}, à la troisième la \emph{religion}, à la quatrième l'\emph{anthropologie}. Mais au fond, on pourrait tout ramener à l'anthropologie, puisque les trois premières questions se rapportent à la dernière.}

  \chunks{\begin{flushright}
          \large\textsc{Kant}, \textit{Logik}.
          \end{flushright}}
         {\begin{flushright}
          \large\textsc{Kant}, \textit{Logique}.
          \end{flushright}}

\end{paracol}

\end{document}
