
\documentclass[a4paper,12pt]{article}

% https://de.wikisource.org/wiki/Das_Judenthum_in_der_Musik_(1869)
% traduction de B. de Trèves
% https://gallica.bnf.fr/ark:/12148/bpt6k5469256g
% https://fr.wikisource.org/wiki/Livre:Wagner_-_Le_Juda%C3%AFsme_dans_la_musique,_trad._Tr%C3%A8ves.djvu

\usepackage[a4paper,margin=8mm,bmargin=16mm]{geometry}
\usepackage[german,main=french]{babel}
\usepackage{fontspec}
\usepackage{microtype}

% https://texnique.fr/osqa/questions/8423/lualatex-et-ligatures
%\setmainfont{EBGaramond}
%[
%  Extension      = .otf,
%  UprightFont    = *-Regular,
%  ItalicFont     = *-Italic,
%  BoldFont       = *-Bold,
%  BoldItalicFont = *-BoldItalic,
%  Numbers        = Lowercase,
%  Ligatures      = Discretionary,
%  Style          = Swash
%]

\setmainfont{Alegreya}

\usepackage{paracol}
\newcommand\chunks[2]{%
\begin{leftcolumn*}\begin{otherlanguage}{german}%
{#1}%
\end{otherlanguage}\end{leftcolumn*}%
\begin{rightcolumn}\begin{otherlanguage}{french}%
{#2}%
\end{otherlanguage}\end{rightcolumn}%
}
% https://tex.stackexchange.com/a/505562/295527

\usepackage{luacolor}
\usepackage[soul]{lua-ul}
% https://tex.stackexchange.com/a/747323
%\renewcommand{\hl}{\relax}
%\renewcommand{\hl}[1]{#1}

%\begin{center}
%  $\ast$~$\ast$~$\ast$
%\end{center}
\usepackage{psvectorian}
\newcommand{\ornamentbreak}{%
  \begin{center}
    \psvectorian[width=2em]{2}\quad\psvectorian[width=2em,mirror]{2}%
  \end{center}%
}

\begin{document}

\thispagestyle{empty}
\setlength{\columnsep}{8mm}
\setlength{\columnseprule}{0.4pt}

\begin{paracol}{2}
  \chunks{\section*{Das Judenthum in der Musik}}
         {\section*{Le Judaïsme dans la musique}}

  \chunks{In der „Neuen Zeitschrift für Musik“ kam unlängst ein „hebräischer Kunstgeschmack“ zur Sprache: eine Anfechtung und eine Vertheidigung dieses Ausdruckes konnten und durften nicht ausbleiben. Es dünkt mich nun nicht unwichtig, den hier zu Grunde liegenden, von der Kritik immer nur noch versteckt oder im Ausbruche einer gewissen Erregtheit berührten Gegenstand näher zu erörtern. Hierbei wird es nicht darauf ankommen, etwas Neues zu sagen, sondern die unbewußte Empfindung, die sich im Volke als innerlichste Abneigung gegen jüdisches Wesen kundgiebt, zu erklären, somit etwas wirklich Vorhandenes deutlich auszusprechen, keinesweges aber etwas Unwirkliches durch die Kraft irgendwelcher Einbildung künstlich beleben zu wollen. Die Kritik verfährt wider ihre Natur, wenn sie in Angriff oder Vertheidigung etwas Anderes will.}
         {Dernièrement, il a été question dans la « Neue Zeitschrift für Musik » de « goût artistique hébraïque ». Il est impossible qu’une pareille assertion restât sans provoquer une attaque, puis une défense. Il m’apparaît comme suffisamment important de traiter ce sujet plus à fond que ne l’a fait jusqu’à présent la critique qui ne s’en est occupée que d’une manière indirecte ou sous l’empire de la passion. Il ne sera pas question ici de trouver quelque chose de nouveau, mais d’expliquer plutôt l’impression inconsciente de recul moral que provoque, chez le peuple, la manière d’être des Juifs ; il ne s’agira pas non plus de faire vivre une idée par la force créatrice de notre imagination, mais de noter une réalité bien existante. La critique irait à l’encontre de son devoir si elle faisait autrement.}
  \chunks{Da wir den Grund der volksthümlichen Abneigung auch unsrer Zeit gegen jüdisches Wesen uns hier lediglich in Bezug auf die Kunst, und namentlich die Musik, erklären wollen, haben wir der Erläuterung derselben Erscheinung auf dem Felde der Religion und Politik gänzlich vorüberzugehen. In der Religion sind uns die Juden längst keine hassenswürdigen Feinde mehr, – Dank allen Denen, welche innerhalb der christlichen Religion selbst den Volkshaß auf sich gezogen haben! In der reinen Politik sind wir mit den Juden nie in wirklichen Conflict gerathen; wir gönnten ihnen selbst die Errichtung eines jerusalemischen Reiches, und hatten in dieser Beziehung eher zu bedauern, daß Herr v. Rothschild zu geistreich war, um sich zum König der Juden zu machen, wogegen er bekanntlich es vorzog, "`der Jude der Könige"' zu bleiben.}% ''text'' is not the standard way, ``text'' is. But for German I recommend using babel's shorthands "`text"'. https://tex.stackexchange.com/q/150945/295527
         {Nous passerons sous silence l’antagonisme qui peut exister entre les Juifs et nous dans le domaine de la religion ou de la politique, car notre but est de \hl{ne considérer la chose que sous le rapport de l’art et plus spécialement de la musique}. D’ailleurs, au point de vue religieux, les Juifs ne peuvent plus être considérés comme étant nos ennemis et cela grâce à ceux qui, au sein même de notre religion, se sont attiré la haine populaire. En ce qui concerne la question politique, nous ne sommes jamais entrés en conflit direct avec les Juifs ; nous avons souhaité qu’il se crée un jour un royaume juif à Jérusalem, et de ce côté-là nos regrets furent réels, quand nous vîmes que M. de Rotschild préféra, en homme d’esprit qu’il est, rester le « Juif des Rois » plutôt que de devenir le « Roi des Juifs ».}
  \chunks{Anders verhält es sich da, wo die Politik zur Frage der Gesellschaft wird: hier hat uns die Sonderstellung der Juden seit ebenso lange als Aufforderung zu menschlicher Gerechtigkeitsübung gegolten, als in uns selbst der Drang nach socialer Befreiung zu deutlicherem Bewußtsein erwachte. Als wir für Emancipation der Juden stritten, waren wir aber doch eigentlich mehr Kämpfer für ein abstractes Princip, als für den concreten Fall: wie all unser Liberalismus ein nicht sehr hellsehendes Geistesspiel war, indem wir für die Freiheit des Volkes uns ergingen, ohne Kenntniß dieses Volkes, ja mit Abneigung gegen jede wirkliche Berührung mit ihm, so entsprang auch unser Eifer für die Gleichberechtigung der Juden viel mehr aus der Anregung eines allgemeinen Gedankens, als aus einer realen Sympathie; denn bei allem Reden und Schreiben für Judenemancipation fühlten wir uns bei wirklicher, thätiger Berührung mit Juden von diesen stets unwillkürlich abgestoßen.}
         {Certes, là où la question politique se mue en question sociale, il n’en est plus de même ; notre sentiment inné de justice humaine nous a fait un devoir de voler au secours des Juifs persécutés et cela d’autant plus que s’éveillait en nous-mêmes une aspiration vers la liberté et l’indépendance de la société. Toutefois, même \hl{lorsque nous luttions pour l’émancipation des Juifs}, nous étions davantage les défenseurs d’\hl{un principe abstrait}, que celui d’un cas concret bien déterminé. Par le fait même que tout notre libéralisme était plutôt entaché de trouble, nous défendions le peuple juif sans même chercher à le connaître – et même en faisant tout pour l’éviter – et nous devons reconnaître que notre zèle à exiger l’égalité des droits pour les Juifs, avait davantage sa source dans un état de surexcitation, bien plus que dans une sympathie raisonnée ou réelle. Car, malgré toutes nos paroles et nos écrits en faveur de l’émancipation des Juifs, nous ne pouvions, à l’approche de ceux-ci, nous empêcher de témoigner \hl{une involontaire aversion}.}
  \chunks{Hier treffen wir denn auf den Punkt, der unsrem Vorhaben uns näher bringt: wir haben uns das unwillkürlich Abstoßende, welches die Persönlichkeit und das Wesen der Juden für uns hat, zu erklären, um diese instinctmäßige Abneigung zu rechtfertigen, von welcher wir doch deutlich erkennen, daß sie stärker und überwiegender ist, als unser bewußter Eifer, dieser Abneigung uns zu entledigen.}
         {Nous touchons ici le point capital de notre sujet. Il nous faut expliquer le pourquoi de cette répulsion involontaire que provoquent en nous les Juifs, et tenter de justifier \hl{cette antipathie qui reste, en fin de compte, plus forte en notre esprit que la tentation que nous avons de nous en libérer}.}
  \chunks{Noch jetzt belügen wir uns in dieser Beziehung nur absichtlich, wenn wir es für verpönt und unsittlich halten zu müssen glauben, unsren natürlichen Widerwillen gegen jüdisches Wesen öffentlich kundzugeben. Erst in neuester Zeit scheinen wir zu der Einsicht zu gelangen, daß es vernünftiger sei, von dem Zwange jener Selbsttäuschung uns frei zu machen, um dafür ganz nüchtern den Gegenstand unsrer gewaltsamen Sympathie zu betrachten, und unsren, trotz aller liberalen Vorspiegelungen bestehenden, Widerwillen gegen ihn uns zum Verständniß zu bringen.}
         {C’est encore nous illusionner en connaissance de cause, lorsque nous croyons immoral et contraire aux lois établies, que de nous laisser aller à notre aversion naturelle pour l’esprit juif. Il n’y a que fort peu de temps que nous avons compris qu’il serait plus raisonnable de nous libérer de nos suggestions afin d’examiner, dans le calme, l’objet de notre puissante sympathie afin que nous arrivions à comprendre cette aversion que nous avons pour lui, en dépit de nos illusions les plus libérales.}
  \chunks{Wir gewahren nun zu unsrem Erstaunen, daß wir bei unsrem liberalen Kampfe in der Luft schwebten und mit Wolken fochten, während der schöne Boden der ganz realen Wirklichkeit einen Aneigner fand, den unsre Luftsprünge zwar sehr wohl unterhielten, der uns aber doch für viel zu albern hält, um hierfür uns durch einiges Ablassen von diesem usurpirten realen Boden zu entschädigen. Ganz unvermerkt ist der ‚Gläubiger der Könige‘ zum König der Gläubigen geworden, und wir können nun die Bitte dieses Königs um Emancipirung nicht anders als ungemein naiv finden, da wir vielmehr uns in die Nothwendigkeit versetzt sehen, um Emancipirung von den Juden zu kämpfen.}
         {À notre profonde stupeur, nous fîmes alors la découverte que dans nos luttes libérales nous planions très haut dans le ciel, que nous nous battions avec les nuées, pendant que \hl{la terre, cette réalité tangible}, était accaparée par un possesseur qui daignait peut-être prendre goût à nos sauts périlleux, mais qui avait surtout en vue de ne point nous dédommager de notre intervention et demeurait bel et bien l’usurpateur de nos biens. Petit à petit, le « créancier des rois » est devenu « le roi des croyants » et nous serions vraiment naïfs d’écouter sa demande d’émancipation, alors qu’en fait, c’est nous qui sommes réduits à nous libérer de la tutelle juive.}
  \chunks{Der Jude ist nach dem gegenwärtigen Stande der Dinge dieser Welt wirklich bereits mehr als emancipirt: er herrscht, und wird so lange herrschen, als das Geld die Macht bleibt, vor welcher all unser Thun und Treiben seine Kraft verliert. Daß das geschichtliche Elend der Juden und die räuberische Roheit der christlich-germanischen Gewalthaber den Söhnen Israels diese Macht selbst in die Hände geführt haben, braucht hier nicht erst erörtert zu werden. Daß aber die Unmöglichkeit, auf Grundlage derjenigen Stufe, auf welche jetzt die Entwickelung der Künste gelangt ist, ohne gänzliche Veränderung dieser Grundlage Natürliches, Nothwendiges und wahrhaft Schönes weiter zu bilden, den Juden auch den öffentlichen Kunstgeschmack unsrer Zeit zwischen die geschäftigen Finger gebracht hat, davon haben wir die Gründe hier etwas näher zu betrachten.}
         {Dans l’état de choses actuel, le Juif est en fait plus qu’émancipé ; \hl{il règne} et règnera aussi longtemps que l’Argent restera la Force devant laquelle se briseront tout notre travail et tous nos efforts. Ce n’est pas le lieu de rechercher ici comment cette force a été transmise aux mains des Juifs, en dépit du sort malheureux du peuple israélite et de la rudesse des conquérants germano-chrétiens. Par contre, nous devons montrer \hl{comment le goût artistique public a été livré aux mains mercantiles des Juifs}, et l’impossibilité qui existe de nos jours – à moins d’un bouleversement radical – à ce qu’on puisse créer en art quelque chose de naturellement et nécessairement beau.}
  \chunks{Was den Herren der römischen und mittelalterlichen Welt der leibeigene Mensch in Plack und Jammer gezinst hat, das setzt heut’ zu Tage der Jude in Geld um: wer merkt es den unschuldig aussehenden Papierchen an, daß das Blut zahlloser Geschlechter an ihnen klebt? Was die Heroen der Künste dem kunstfeindlichen Dämon zweier unseliger Jahrtausende mit unerhörter, Luft und Leben verzehrender Anstrengung abrangen, setzt heute der Jude in Kunstwaarenwechsel um: wer sieht es den manierlichen Kunststückchen an, daß sie mit dem heiligen Nothschweiße des Genies zweier Jahrtausende geleimt sind?}
         {Ce que l’homme des temps du moyen âge ou de la Rome antique payait à ses maîtres par des tortures et des tourments, le Juif l’a converti aujourd’hui en monnaie. Mais qui remarque que ces chiffons de papier aux allures innocentes sont rouges du sang de nombreuses générations ? \hl{Ce que les héros de l’art surent arracher au démon ennemi des arts}, en des siècles et des siècles de luttes, d’efforts et d’infortunes, \hl{le Juif en fait aujourd’hui un article commercial}. Qui songe que les productions artistiques contemporaines, si maniérées soient-elles, sont l’aboutissant de vingt siècles d’efforts incessants ?}
  \chunks{Wir haben nicht erst nöthig, die Verjüdung der modernen Kunst zu bestätigen; sie springt in die Augen und bestätigt sich den Sinnen von selbst. Viel zu weit ausholend würden wir auch verfahren müssen, wollten wir aus dem Charakter unsrer Kunstgeschichte selbst diese Erscheinung nachweislich zu erklären unternehmen. Dünkt uns aber das Nothwendigste die Emancipation von dem Drucke des Judenthumes, so müssen wir es vor Allem für wichtig erachten, unsre Kräfte zu diesem Befreiungskampfe zu prüfen. Diese Kräfte gewinnen wir aber nun nicht aus einer abstracten Definition jener Erscheinung selbst, sondern aus dem genauen Bekanntwerden mit der Natur der uns innewohnenden unwillkürlichen Empfindung, die sich uns als instinctmäßiger Widerwille gegen das jüdische Wesen äußert: an ihr, der unbesieglichen, muß es uns, wenn wir sie ganz unumwunden eingestehen, deutlich werden, was wir an jenem Wesen hassen; was wir dann bestimmt kennen, dem können wir die Spitze bieten; ja schon durch seine nackte Aufdeckung dürfen wir hoffen, den Dämon aus dem Felde zu schlagen, auf dem er sich nur im Schutze eines dämmerigen Halbdunkels zu halten vermag, eines Dunkels, das wir gutmüthigen Humanisten selbst über ihn warfen, um uns seinen Anblick minder widerwärtig zu machen.\ornamentbreak}
         {Pas n’est besoin de prouver \hl{la judaïsation de l’art moderne} ; elle ne fait pas de doute. Il serait nécessaire de remonter très loin dans l’histoire de notre art si nous voulions en rechercher les sources. Toutefois, si ce qui nous apparaît le plus nécessaire à cette heure consiste à extirper cette tutelle juive, il faut commencer par \hl{reconnaître nos forces} en vue de cette libération. Mais cette force, nous ne l’obtiendrons point d’une définition abstraite, mais bien d’une connaissance précise de \hl{ce sentiment d’aversion qui existe en nous} et qui nous éloigne de tout ce qui est juif. Cette répugnance naturelle nous fournira les bases rationnelles de ce qui nous choque dans l’élément judaïque. Nous pourrons alors opposer notre pointe à ce que nous connaîtrons avec précision. Et nous réussirons ainsi à chasser le Démon du champ où il se cache dans \hl{une demi-obscurité} qui est due en grande partie au zèle que nous autres, humanitaires, avions mis à l’entourer pour nous rendre sa vue moins blessante.}
% * * *
  \chunks{Der Jude, der bekanntlich einen Gott ganz für sich hat, fällt uns im gemeinen Leben zunächst durch seine äußere Erscheinung auf, die, gleichviel welcher europäischen Nationalität wir angehören, etwas dieser Nationalität unangenehm Fremdartiges hat: wir wünschen unwillkürlich mit einem so aussehenden Menschen Nichts gemein zu haben. Dies mußte bisher als ein Unglück für den Juden gelten; in neuerer Zeit erkennen wir aber, daß er bei diesem Unglücke sich ganz wohl fühlt; nach seinen Erfolgen darf ihn seine Unterschiedenheit von uns als eine Auszeichnung dünken. Der moralischen Seite in der Wirkung dieses an sich unangenehmen Naturspieles vorübergehend, wollen wir hier nur auf die Kunst bezüglich erwähnen, daß dieses Aeußere uns nie als ein Gegenstand der darstellenden Kunst denkbar sein kann: wenn die bildende Kunst Juden darstellen will, nimmt sie ihre Modelle meist aus der Phantasie, mit weislicher Veredelung oder gänzlicher Hinweglassung alles dessen, was uns im gemeinen Leben die jüdische Erscheinung eben charakterisirt. Nie verirrt sich der Jude aber auf die theatralische Bühne: die Ausnahmen hiervon sind der Zahl und der Besonderheit nach von der Art, daß sie die allgemeine Annahme nur bestätigen. Wir können uns auf der Bühne keinen antiken oder modernen Charakter, sei es ein Held oder ein Liebender, von einem Juden dargestellt denken, ohne unwillkürlich das bis zur Lächerlichkeit Ungeeignete einer solchen Vorstellung zu empfinden.[1] Dies ist sehr wichtig: einen Menschen, dessen Erscheinung wir zu künstlerischer Kundgebung, nicht in dieser oder jener Persönlichkeit, sondern allgemeinhin seiner Gattung nach, für unfähig halten müssen, dürfen wir zur künstlerischen Aeußerung seines Wesens überhaupt ebenfalls nicht für befähigt halten.}
         {Le Juif, qui a un Dieu bien à lui, nous frappe à première vue par son aspect extérieur, et cela à quelque nationalité qu’il appartienne et \hl{nous nous sentons, de ce fait devant un étranger}. Involontairement, nous désirons n’avoir rien de commun avec un pareil homme. Jusqu’à cette heure, cette particularité passait pour une disgrâce pour le Juif ; nous avons dû reconnaître, à présent, qu’il s’en accommodait fort bien ; après les succès qu’il a remportés, tout ce qui le distingue de nous est pour lui un avantage de plus. Sans vouloir insister sur le côté moral de cette disgrâce physique, nous devons dire toutefois que cet extérieur purement d’ordre matériel ne saurait être reproduit par l’art. Quand la sculpture veut représenter un Juif, elle prend son modèle dans le domaine de l’imagination, en s’efforçant d’idéaliser la chose ou en lui enlevant tout ce qui constitue justement à nos yeux le type du Juif. Jamais le Juif ne paraîtra sur une scène théâtrale ; les exceptions à cette règle sont si rares qu’elles ne font que la confirmer. Nous ne pouvons en effet pas nous figurer un Juif représentant sur la scène tel héros ou tel amoureux, ancien ou moderne, sans qu’aussitôt nous apparaisse l’énormité et le ridicule d’une telle impropriété[1]. Et ceci est très important : nous nous trouvons dans l’impossibilité de comprendre qu’un individu puisse réaliser une manifestation artistique, et cela non à cause de tel ou tel personnage, mais parce qu’il est représentatif d’une race qui est contraire à notre propre idéal.}
  \chunks{Ungleich wichtiger, ja entscheidend wichtig ist jedoch die Beachtung der Wirkung auf uns, welche der Jude durch seine Sprache hervorbringt; und namentlich ist dies der wesentliche Anhaltspunkt für die Ergründung des jüdischen Einflusses auf die Musik. – Der Jude spricht die Sprache der Nation, unter welcher er von Geschlecht zu Geschlecht lebt, aber er spricht sie immer als Ausländer. Wie es von hier abliegt, uns mit den Gründen auch dieser Erscheinung zu befassen, dürfen wir ebenso die Anklage der christlichen Civilisation unterlassen, welche den Juden in seiner gewaltsamen Absonderung erhielt, als wir andererseits durch die Berührung der Erfolge dieser Absonderung die Juden auch keinesweges zu bezichtigen im Sinne haben können. Dagegen liegt es uns hier ob, den ästhetischen Charakter dieser Ergebnisse zu beleuchten.}
         {Une chose qui mérite toute notre attention est \hl{l'effet que produit sur nous le Juif par sa parole, par son langage}, et c’est de ce point de vue qu’il faut considérer l’influence qu’exerce le Juif sur la musique. Le Juif parle la langue de la nation où il vit de génération en génération, mais il la parle comme un étranger. Ce qui nous intéresse n’étant pas de chercher les causes de cet état de choses, nous n’avons pas à \hl{prendre parti contre la civilisation chrétienne, et à lui reprocher d’avoir condamné le Juif à sa farouche solitude}, et nous n’avons pas à nous occuper des résultats de cet isolement. Qu’il nous suffise de déduire et de mettre en lumière les caractères esthétiques de ces événements.}
  \chunks{Zunächst muß im Allgemeinen der Umstand, daß der Jude die modernen europäischen Sprachen nur wie erlernte, nicht als angeborene Sprachen redet, ihn von aller Fähigkeit, in ihnen sich seinem Wesen entsprechend, eigenthümlich und selbständig kundzugeben, ausschließen. Eine Sprache, ihr Ausdruck und ihre Fortbildung, ist nicht das Werk Einzelner, sondern einer geschichtlichen Gemeinsamkeit: nur wer unbewußt in dieser Gemeinsamkeit aufgewachsen ist, nimmt auch an ihren Schöpfungen theil. Der Jude stand aber außerhalb einer solchen Gemeinsamkeit, einsam mit seinem Jehova in einem zersplitterten, bodenlosen Volksstamme, welchem alle Entwickelung aus sich versagt bleiben mußte, wie selbst die eigenthümliche (hebräische) Sprache dieses Stammes ihm nur als eine todte erhalten ist. In einer fremden Sprache wahrhaft zu dichten, ist nun bisher selbst den größten Genies noch unmöglich gewesen. Unsre ganze europäische Civilisation und Kunst ist aber für den Juden eine fremde Sprache geblieben; denn, wie an der Ausbildung dieser, hat er auch an der Entwickelung jener nicht theilgenommen, sondern kalt, ja feindselig hat der Unglückliche, Heimathlose ihr höchstens nur zugesehen. In dieser Sprache, dieser Kunst kann der Jude nur nachsprechen, nachkünsteln, nicht wirklich redend dichten oder Kunstwerke schaffen.}
         {Par le fait que le Juif parle les langues européennes, non comme un indigène, mais comme des langues apprises, il est dans l’incapacité de s’exprimer suivant l’originalité et le génie propres à chaque langue. Une langue, son expression et son développement, n’est pas le fait de quelques individus, mais doit son existence \hl{au travail de toute une communauté}, et seul celui qui fait partie intégrante de cette communauté, peut prendre part à ce travail continu de création. Mais le Juif se tenait forcément en dehors de cette communauté, seul avec son Jéhovah, faisant partie d’une race dispersée et dont la langue elle-même (l’hébreu) devait rester une langue morte. Jusqu’à présent, il a été \hl{impossible aux plus grands génies de s’exprimer en poète dans une autre langue que leur langue maternelle} ; et pour le Juif, toute la civilisation et tout l’art européens sont restés choses étrangères, car il n’a pas plus participé à la formation de la première qu’au développement du deuxième et il est le plus souvent resté un spectateur froid, sinon hostile. Le Juif ne peut donc, dans cette langue, faire œuvre de poète ou d’artiste : il devra se contenter \hl{d’imiter, de répéter}.}
  \chunks{Im Besonderen widert uns nun aber die rein sinnliche Kundgebung der jüdischen Sprache an. Es hat der Cultur nicht gelingen wollen, die sonderliche Hartnäckigkeit des jüdischen Naturells in Bezug auf Eigenthümlichkeiten der semitischen Aussprechweise durch zweitausendjährigen Verkehr mit europäischen Nationen zu brechen. Als durchaus fremdartig und unangenehm fällt unsrem Ohre zunächst ein zischender, schrillender, summsender und murksender Lautausdruck der jüdischen Sprechweise auf: eine unsrer nationalen Sprache gänzlich uneigenthümliche Verwendung und willkürliche Verdrehung der Worte und der Phrasenconstructionen giebt diesem Lautausdrucke vollends noch den Charakter eines unerträglich verwirrten Geplappers, bei dessen Anhörung unsre Aufmerksamkeit unwillkürlich mehr bei diesem widerlichen Wie, als bei dem darin enthaltenen Was der jüdischen Rede verweilt.}
         {D’ailleurs, l’accent purement physique du parler juif nous choque désagréablement. Malgré un contact de plus de vingt siècles avec les nations européennes, la civilisation n’est pas arrivée à faire disparaître certaines particularités d’expression et de tonalités propres au Juif. Rien de plus désagréable pour notre oreille que le son à la fois zézayant, criard et traînard qui est à la base de la prononciation juive. De plus, le Juif a \hl{une façon tout à fait impropre d’employer notre langue}, l’altération systématique qu’il fait de certains termes, certains tours de phrases qu’il emploie mal à propos ne manquent pas de provoquer en notre esprit un trouble tel que nous ne demandons plus ce qu’il nous dit, mais comment il nous le dit.}
  \chunks{Wie ausnehmend wichtig dieser Umstand zur Erklärung des Eindruckes namentlich der Musikwerke moderner Juden auf uns ist, muß vor Allem erkannt und festgehalten werden. Hören wir einen Juden sprechen, so verletzt uns unbewußt aller Mangel rein menschlichen Ausdruckes in seiner Rede: die kalte Gleichgiltigkeit des eigenthümlichen „Gelabbers“ in ihr steigert sich bei keiner Veranlassung zur Erregtheit höherer, herzdurchglüheter Leidenschaft. Sehen wir uns dagegen im Gespräche mit einem Juden zu diesem erregteren Ausdrucke gedrängt, so wird er uns stets ausweichen, weil er zur Erwiderung unfähig ist. Nie erregt sich der Jude im gemeinsamen Austausche der Empfindungen mit uns, sondern, uns gegenüber, nur im ganz besonderen egoistischen Interesse seiner Eitelkeit oder seines Vortheils, was solcher Erregtheit, bei dem entstellenden Ausdrucke seiner Sprechweise überhaupt, dann immer den Charakter des Lächerlichen giebt, und uns Alles, nur nicht Sympathie für des Redenden Interesse zu erwecken vermag. Muß es uns schon denkbar erscheinen, daß bei gemeinschaftlichen Anliegenheiten unter einander, und namentlich da, wo in der Familie die rein menschliche Empfindung zum Durchbruche kommt, gewiß auch Juden ihren Gefühlen einen Ausdruck zu geben vermögen, der für sie gegenseitig von entsprechender Wirkung ist, so kann das doch hier nicht in Betrachtung kommen, wo wir den Juden zu vernehmen haben, der im Lebens- und Kunstverkehr geradesweges zu uns spricht.}
         {Il est de toute importance de reconnaître ce phénomène pour nous expliquer l’impression produite sur nous par les œuvres musicales des Juifs modernes. Lorsque nous entendons un Juif parler, nous sommes blessés de rencontrer dans ses discours une absence complète de chaleur et d’humanité. \hl{La froide indifférence de son bredouillement ne tend jamais à s’inspirer des accents véhéments de la passion.} Que nous nous sentions, dans une conversation avec un Juif, poussés à nous émouvoir, nous verrons celui-ci se dérober, car il est incapable de répondre sur le même ton. En aucun cas, le Juif ne s’anime dans un échange d’impressions, à moins que ne soient en jeu ou sa vanité ou ses intérêts ; et alors son animation est telle, que grâce à l’accent caricatural de son langage, il ne produit sur son interlocuteur qu’un effet de ridicule qui est loin de faire naître en nous quelque sympathie pour lui. Certes, il nous faut admettre que dans leurs rapports entre eux et plus particulièrement dans leurs rapports familiaux, les Juifs arrivent à donner à leurs sentiments des expressions plus touchantes, mais nous ne pouvons en tenir compte ici, où il s’agit uniquement des relations que les Juifs entretiennent avec et sur nous par le commerce de la vie et des arts.}
  \chunks{Macht nun die hier dargethane Eigenschaft seiner Sprechweise den Juden fast unfähig zur künstlerischen Kundgebung seiner Gefühle und Anschauungen durch die Rede, so muß zu solcher Kundgebung durch den Gesang seine Befähigung noch bei weitem weniger möglich sein. Der Gesang ist eben die in höchster Leidenschaft erregte Rede: die Musik ist die Sprache der Leidenschaft. Steigert der Jude seine Sprechweise, in der er sich uns nur mit lächerlich wirkender Leidenschaftlichkeit, nie aber mit sympathisch berührender Leidenschaft zu erkennen geben kann, gar zum Gesang, so wird er uns damit geradesweges unausstehlich. Alles, was in seiner äußeren Erscheinung und seiner Sprache uns abstoßend berührte, wirkt in seinem Gesange auf uns endlich davonjagend, so lange wir nicht durch die vollendete Lächerlichkeit dieser Erscheinung gefesselt werden sollten. Sehr natürlich geräth im Gesange, als dem lebhaftesten und unwiderleglich wahrsten Ausdrucke des persönlichen Empfindungswesens, die für uns widerliche Besonnenheit der jüdischen Natur auf ihre Spitze, und auf jedem Gebiete der Kunst, nur nicht auf demjenigen, dessen Grundlage der Gesang ist, sollten wir, einer natürlichen Annahme gemäß, den Juden je für kunstbefähigt halten dürfen.}
         {Il est donc notoire que si le Juif, ainsi que nous l’avons démontré, est incapable d’exprimer par son langage des sentiments et des idées au moyen du discours, il pourra encore moins les manifester au moyen du chant, \hl{le chant étant à proprement parler, le discours porté à son paroxysme, car la musique est par excellence le langage de la passion}. Si, d’aventure, le Juif cherche à élever l’animation de son verbiage jusqu’au chant, il lui sera impossible de nous émouvoir par une excitation feinte et foncièrement ridicule et il se rendra ainsi d’autant plus insupportable. Et tout ce qui nous froissait déjà dans son langage et son physique ne réussirait, alors qu’il chante, qu’à nous mettre en fuite, si la bouffonnerie de ce spectacle ne nous retenait. Si l’on considère le chant comme l’expression la plus adéquate d’une sensibilité exagérée, mais profondément humaine, il est naturel que le Juif atteigne le plus haut degré de sa sécheresse égoïste et nous pouvons donc en déduire que dans tous les domaines de la vie artistique – en dehors de ceux mêmes qui ont le chant à leur base — on est en droit de dénier à la race juive toute possibilité d’exprimer des pensées d’art.}
  \chunks{Die sinnliche Anschauungsgabe der Juden ist nie vermögend gewesen, bildende Künstler aus ihnen hervorgehen zu lassen: ihr Auge hat sich von je mit viel praktischeren Dingen befaßt, als da Schönheit und geistiger Gehalt der förmlichen Erscheinungswelt sind. Von einem jüdischen Architekten oder Bildhauer kennen wir in unsren Zeiten, meines Wissens, Nichts: ob neuere Maler jüdischer Abkunft in ihrer Kunst wirklich geschaffen haben, muß ich Kennern von Fach zur Beurtheilung überlassen; sehr vermuthlich dürften aber diese Künstler zur bildenden Kunst keine andere Stellung einnehmen, als diejenige der modernen jüdischen Componisten zur Musik, zu deren genauerer Beleuchtung wir uns nun wenden.}
         {C’est à leur faculté de concrétion des choses [traduction à revoir] que les Juifs doivent de n’avoir jamais eu d’artistes créateurs de formes plastiques. Leur œil est trop exercé à des choses plus pratiques, pour qu’il ait pu s’arrêter à la conception d’une beauté idéale des formes et des lignes. Pour mon compte, je n’ai pas connaissance qu’il ait jamais existé d’architectes ou de sculpteurs juifs. Et en ce qui concerne les peintres de race ou d’origine juive, il reste à savoir si ce sont réellement des créateurs, ou s’ils n’occupent dans ce domaine des arts que le même rang que les compositeurs juifs modernes, rang que nous allons préciser de notre mieux.}
  \chunks{Der Jude, der an sich unfähig ist, weder durch seine äußere Erscheinung, noch durch seine Sprache, am allerwenigsten aber durch seinen Gesang, sich uns künstlerisch kundzugeben, hat nichtsdestoweniger es vermocht, in der verbreitetsten der modernen Kunstarten, der Musik, zur Beherrschung des öffentlichen Geschmackes zu gelangen. – Betrachten wir, um uns diese Erscheinung zu erklären, zunächst, wie es dem Juden möglich ward, Musiker zu werden.}
         {Bien qu’il fût impossible au Juif de passer à nos yeux pour un artiste, tant à cause de son aspect extérieur que de son langage, et plus encore de son chant, il n’en a pas moins réussi à s’implanter dans une forme artistique, à vrai dire celle qui est la plus répandue : la musique. Pour nous expliquer ce fait curieux, commençons par nous demander \hl{de quelle façon le Juif a pu devenir musicien}.}
  \chunks{Von der Wendung unsrer gesellschaftlichen Entwickelung an, wo mit immer unumwundenerer Anerkennung das Geld zum wirklich machtgebenden Adel erhoben ward, konnte den Juden, denen Geldgewinn ohne eigentliche Arbeit, d. h. der Wucher, als einziges Gewerbe überlassen worden war, das Adelsdiplom der neueren, nur noch geldbedürftigen Gesellschaft nicht nur nicht mehr vorenthalten werden, sondern sie brachten es ganz von selbst dahin mit.}
         {Alors que dans les temps passés, on n’avait laissé aux Juifs comme unique profession celle qui ne demandait aucun travail : l’usure, il a été impossible, dans notre société moderne où de plus en plus l’argent est le maître de la puissance et le véritable titre de noblesse, de refuser aux Juifs l’accès des honneurs, d’autant qu’ils apportaient avec leur richesse de quoi se les procurer sans coup férir.}
  \chunks{Unsre moderne Bildung, die nur dem Wohlstande zugänglich ist, blieb ihnen daher um so weniger verschlossen, als sie zu einem käuflichen Luxusartikel herabgesunken war. Von nun an tritt also der gebildete Jude in unsrer Gesellschaft auf, dessen Unterschied vom ungebildeten, gemeinen Juden wir genau zu beachten haben. Der gebildete Jude hat sich die undenklichste Mühe gegeben, alle auffälligen Merkmale seiner niederen Glaubensgenossen von sich abzustreifen: in vielen Fällen hat er es selbst für zweckmäßig gehalten, durch die christliche Taufe auf die Verwischung aller Spuren seiner Abkunft hinzuwirken. Dieser Eifer hat den gebildeten Juden aber nie die erhofften Früchte gewinnen lassen wollen: er hat nur dazu geführt, ihn vollends zu vereinsamen, und ihn zum herzlosesten aller Menschen in einem Grade zu machen, daß wir selbst die frühere Sympathie für das tragische Geschick seines Stammes verlieren mußten.}
         {La civilisation moderne n’étant en fait que l’apanage des gens fortunés et y étant devenue un article de luxe qu’on pouvait se procurer commercialement, les Juifs y eurent facilement accès. À dater de cette époque, nous voyons apparaître dans notre société le Juif cultivé que nous sommes obligés de différencier d’avec le Juif commun et manquant de culture. Le Juif cultivé a épuisé tous ses efforts et s’est donné toutes les peines du monde pour se défaire des signes distinctifs de sa race. Il est même allé jusqu’à se faire baptiser, espérant que le baptême le libérerait de la tare originelle. Il est juste de dire qu’il n’a pas recueilli, pour autant, tout le bénéfice qu’il espérait de son zèle, mais au contraire, cela en a fait l’homme le plus dur, le moins pitoyable qu’on puisse trouver et cela jusqu’au point d’anéantir en nous la sympathie que nous éprouvions pour le destin tragique de sa race.}
  \chunks{Für den Zusammenhang mit seinen ehemaligen Leidensgenossen, den er übermüthig zerriß, blieb es ihm unmöglich einen neuen Zusammenhang mit der Gesellschaft zu finden, zu welcher er sich aufschwang. Er steht nur mit denen in Zusammenhang, welche sein Geld bedürfen: nie hat es aber dem Gelde gelingen wollen, ein gedeihenvolles Band zwischen Menschen zu knüpfen. Fremd und theilnahmlos steht der gebildete Jude inmitten einer Gesellschaft, die er nicht versteht, mit deren Neigungen und Bestrebungen er nicht sympathisirt, deren Geschichte und Entwickelung ihm gleichgiltig geblieben sind. In solcher Stellung haben wir unter den Juden Denker entstehen sehen: der Denker ist der rückwärtsschauende Dichter; der wahre Dichter ist aber der vorverkündende Prophet. Zu solchem Prophetenamte befähigt nur die tiefste, seelenvollste Sympathie mit einer großen, gleichstrebenden Gemeinsamkeit, deren unbewußten Ausdruck der Dichter eben nach seinem Inhalte deutet.}
         {N’ayant plus de lien avec ses anciens compagnons de douleur puisqu’il l’avait orgueilleusement rejeté loin de lui, il fut impossible au Juif cultivé de s’en créer de nouveaux dans la société où il se hissait. Seuls, ceux qui ont besoin de son argent sont en communion avec lui ; mais jamais l’argent n’a créé de liens durables entre les hommes. Le Juif cultivé se dresse donc isolé et indifférent dans une société qu’il ne comprend pas, de laquelle il ne partage ni les sympathies ni les penchants et dont l’histoire et l’évolution lui sont restées étrangères. Il a surgi néanmoins parmi ces Juifs quelques penseurs ; mais le penseur est un poète qui regarde derrière lui. Le vrai poète est seul celui qui est un prophète et dont les regards sont fixés sur l’avenir. Pour être un tel poète, il faut nourrir l’affection la plus ardente pour une communauté dont le poète partage les sentiments et qui s’exprime naturellement par son génie personnel.}
  \chunks{Von dieser Gemeinsamkeit der Natur seiner Stellung nach gänzlich ausgeschlossen, aus dem Zusammenhange mit seinem eigenen Stamme gänzlich herausgerissen, konnte dem vornehmeren Juden seine eigene erlernte und bezahlte Bildung nur als Luxus gelten, da er im Grunde nicht wußte, was er damit anfangen sollte. Ein Theil dieser Bildung waren nun aber auch unsre modernen Künste geworden, und unter diesen namentlich diejenige Kunst, die sich am leichtesten eben erlernen läßt, die Musik, und zwar die Musik, die, getrennt von ihren Schwesterkünsten, durch den Drang und die Kraft der größten Genies auf die Stufe allgemeinster Ausdrucksfähigkeit erhoben worden war, auf welcher sie nun entweder, im neuen Zusammenhange mit den anderen Künsten, das Erhabenste, oder, bei fortgesetzter Trennung von jenen, nach Belieben auch das Allergleichgiltigste und Trivialste aussprechen konnte. Was der gebildete Jude in seiner bezeichneten Stellung auszusprechen hatte, wenn er künstlerisch sich kundgeben wollte, konnte natürlich eben nur das Gleichgiltige und Triviale sein, weil sein ganzer Trieb zur Kunst ja nur ein luxuriöser, unnöthiger war. Jenachdem seine Laune, oder ein außerhalb der Kunst liegendes Interesse es ihm eingab, konnte er so, oder auch anders sich äußern; denn nie drängte es ihn, ein Bestimmtes, Nothwendiges und Wirkliches auszusprechen; sondern er wollte gerade eben nur sprechen, gleichviel was, so daß ihm natürlich nur das Wie als besorgenswerthes Moment übrig blieb.}
         {Le Juif cultivé n’ayant plus de lien commun avec sa race, ne pouvant pas avoir avec la société où il vit, n’eut bientôt que faire de sa culture étudiée et payée, qui n’était plus pour lui qu’un article de luxe. Or, dans cette culture acquise figurent les arts modernes et parmi ceux-ci, celui que l’on s’assimile le mieux : la musique. La musique qui, séparée des autres arts, a pu, grâce aux plus grands génies, acquérir une faculté d’expression intense et donner, par une synthèse nouvelle avec les autres arts, une impression touchant au sublime, reste aussi, si elle est isolée, la chose la plus ordinaire et la plus triviale qui soit. Le Juif cultivé, placé dans la situation que nous avons indiquée, ne pourrait donc en voulant donner une interprétation d’art, qu’exprimer des choses triviales et plates, parce que son sentiment artistique n’était somme toute que futilité ou luxe. Il pouvait suivre l’instinct du moment, qui tenait ou d’un caprice ou d’une chose étrangère à l’art. Jamais rien d’impérieux, de nécessaire, de tangible ne l’inspirait. Il n’avait en vue que le besoin de parler, peu lui important de quoi il parlait, n’ayant plus en vue que la façon dont il le dirait.}
  \chunks{Die Möglichkeit, in ihr zu reden, ohne etwas Wirkliches zu sagen, bietet jetzt keine Kunst in so blühender Fülle, als die Musik, weil in ihr die größten Genies bereits das gesagt haben, was in ihr als absoluter Sonderkunst zu sagen war. War dieses einmal ausgesprochen, so konnte in ihr nur noch nachgeplappert werden, und zwar ganz peinlich genau und täuschend ähnlich, wie Papageien menschliche Wörter und Reden nachpapeln, aber ebenso ohne Ausdruck und wirkliche Empfindung, wie diese närrischen Vögel es thun. Nur ist bei dieser nachäffenden Sprache unsrer jüdischen Musikmacher eine besondere Eigenthümlichkeit bemerkbar, und zwar die der jüdischen Sprechweise überhaupt, welche wir oben näher charakterisirten.}
         {Or, comme en musique, en tant qu’art absolu, les plus grands génies ont dit tout ce qu’elle pouvait exprimer, il n’est pas un art où il soit aussi facile de parler pour ne rien dire. Les principes fondamentaux une fois exprimés, on n’arrive plus qu’à faire rabâcher à la façon des perroquets qui répètent les phrases humaines, sans qu’ils y mettent la moindre expression et la moindre chaleur. Et c’est à cette interprétation d’art qu’a abouti en musique le Juif cultivé ; la seule chose qui fasse remarquer plus spécialement son imitation simiesque étant cette élocution juive dont nous avons parlé plus haut.}
  \chunks{Wenn die Eigenthümlichkeiten dieser jüdischen Sprech- und Singweise in ihrer grellsten Sonderlichkeit vor Allem den stammtreu gebliebenen gemeineren Juden zugehören, und der gebildete Jude mit unsäglichster Mühe sich ihrer zu entledigen sucht, so wollen sie doch nichtsdestoweniger mit impertinenter Hartnäckigkeit auch an diesem haften bleiben. Ist dieses Mißgeschick rein physiologisch zu erklären, so erhellt sein Grund aber auch noch aus der berührten gesellschaftlichen Stellung des gebildeten Juden. Mag all unsre Luxuskunst auch fast ganz nur noch in der Luft unsrer willkürlichen Phantasie schweben, eine Faser des Zusammenhanges mit ihrem natürlichen Boden, dem wirklichen Volksgeiste, hält sie doch immer noch nach unten fest. Der wahre Dichter, gleichviel in welcher Kunstart er dichte, gewinnt seine Anregung immer nur noch aus der getreuen, liebevollen Anschauung des unwillkürlichen Lebens, dieses Lebens, das sich ihm nur im Volke zur Erscheinung bringt. Wo findet der gebildete Jude nun dieses Volk? Unmöglich auf dem Boden der Gesellschaft, in welcher er seine Künstlerrolle spielt? Hat er irgend einen Zusammenhang mit dieser Gesellschaft, so ist dies eben nur mit jenem, von ihrem wirklichen, gesunden Stamme gänzlich losgelösten Auswuchse derselben; dieser Zusammenhang ist aber ein durchaus liebloser, und diese Lieblosigkeit muß ihm immer offenbarer werden, wenn er, um Nahrung für sein künstlerisches Schaffen zu gewinnen, auf den Boden dieser Gesellschaft hinabsteigt: nicht nur wird ihm hier Alles fremder und unverständlicher, sondern der unwillkürliche Widerwille des Volkes gegen ihn tritt ihm hier mit verletzendster Nacktheit entgegen, weil er nicht, wie bei den reicheren Classen, durch Berechnung des Vortheils und Beachtung gewisser gemeinschaftlicher Interessen geschwächt oder gebrochen ist. Von der Berührung mit diesem Volke auf das Empfindlichste zurückgestoßen, jedenfalls gänzlich unvermögend, den Geist dieses Volkes zu fassen, sieht sich der gebildete Jude auf die Wurzel seines eigenen Stammes hingedrängt, wo ihm wenigstens das Verständniß unbedingt leichter fällt.}
         {C’est chez le Juif inculte que peuvent le mieux s’observer les caractères particuliers du langage ou du chant juif, mais bien que le Juif cultivé tente de son mieux à s’en dépouiller, il ne peut y réussir. Il doit cette disgrâce à une hérédité d’ordre physiologique ; cependant la situation qu’il occupe dans la société y est également pour beaucoup. Toute expression artistique de luxe a beau planer dans les rêves de notre fantaisie arbitraire, il n’en reste pas moins acquis qu’une fibre la rattache à sa base naturelle qui est l’esprit de la race. Un vrai poète, à quelque art qu’il appartienne, s’inspire toujours de l’observation sympathique de la vie, de cette vie qui se fait jour dans le peuple. Où le Juif cultivé peut-il trouver ce peuple ? Évidemment point dans le milieu où il joue le rôle d’artiste. S’il existe quelque lien entre ce milieu et lui, ce ne peut être de par la force des choses qu’une excroissance détachée de la souche véritable et tôt appelée à disparaître dans l’indifférence générale, dont il se rendra d’autant plus compte qu’il voudra davantage descendre les échelons de cette société pour y chercher des motifs d’émotion artistique. Plus il descendra et plus cette indifférence se muera en une répugnance, car le peuple n’ayant aucun lien social l’unissant à lui, et n’étant pas limité par son intérêt, dans l’expression de ses sentiments, ne craindra pas – comme les classes riches – de témoigner au Juif sa profonde inimitié. Rejeté par ce peuple d’une manière nettement injurieuse, le Juif cultivé se voit contraint de revenir à sa propre race, qu’il saura tout au moins mieux comprendre, qu’il ne l’a fait pour le peuple précité : c’est à cette race qu’il demandera alors une inspiration artistique. Il pensera à cette source mais n’en tirera qu’une façon de faire, et non sujet à traiter.}
  \chunks{Wollend oder nicht wollend, muß er aus diesem Quelle schöpfen; aber nur ein Wie, nicht ein Was hat er ihm zu entnehmen. Der Jude hat nie eine eigene Kunst gehabt, daher nie ein Leben von kunstfähigem Gehalte: ein Gehalt, ein allgemeingiltiger menschlicher Gehalt ist diesem auch jetzt vom Suchenden nicht zu entnehmen, dagegen nur eine sonderliche Ausdrucksweise, und zwar eben diese Ausdrucksweise, welche wir oben näher charakterisirten. Dem jüdischen Tonsetzer bietet sich nun als einziger musikalischer Ausdruck seines Volkes die musikalische Feier seines Jehovadienstes dar: die Synagoge ist der einzige Quell, aus welchem der Jude ihm verständliche volksthümliche Motive für seine Kunst schöpfen kann. Mögen wir diese musikalische Gottesfeier in ihrer ursprünglichen Reinheit auch noch so edel und erhaben uns vorzustellen gesonnen sein, so müssen wir desto bestimmter ersehen, daß diese Reinheit nur in allerwiderwärtigster Trübung auf uns gekommen ist: hier hat sich seit Jahrtausenden Nichts aus innerer Lebensfülle weiterentwickelt, sondern Alles ist, wie im Judenthum überhaupt, in Gehalt und Form starr haften geblieben. Eine Form, welche nie durch Erneuerung des Gehaltes belebt wird, zerfällt aber; ein Ausdruck, dessen Inhalt längst nicht mehr lebendiges Gefühl ist, wird sinnlos und verzerrt sich. Wer hat nicht Gelegenheit gehabt, von der Fratze des gottesdienstlichen Gesanges in einer eigentlichen Volks-Synagoge sich zu überzeugen? Wer ist nicht von der widerwärtigsten Empfindung, gemischt von Grauenhaftigkeit und Lächerlichkeit, ergriffen worden beim Anhören jenes Sinn und Geist verwirrenden Gegurgels, Gejodels und Geplappers, das keine absichtliche Carricatur widerlicher zu entstellen vermag, als es sich hier mit vollem, naivem Ernste darbietet? In der neueren Zeit hat sich der Geist der Reform durch die versuchte Wiederherstellung der älteren Reinheit in diesen Gesängen zwar auch rege gezeigt: was von Seiten der höheren, reflectirenden jüdischen Intelligenz hier geschah, ist aber eben nur ein, seiner Natur nach fruchtloses Bemühen von Oben herab, welches nach Unten nie in dem Grade Wurzel fassen kann, daß dem gebildeten Juden, der eben für seinen Kunstbedarf die eigentliche Quelle des Lebens im Volke aufsucht, der Spiegel seiner intelligenten Bemühungen als diese Quelle entgegenspringen könnte. Er sucht das Unwillkürliche, und nicht das Reflectirte, welches eben sein Product ist; und als dieses Unwillkürliche giebt sich ihm gerade nur jener verzerrte Ausdruck kund.}
         {Il n’existe pas d’art juif, par conséquent point non plus de vie créatrice d’art. Un chercheur ne pourrait pas trouver dans la vie juive, une matière d’art d’ordre général et profondément humain ; il ne se trouverait qu’en face de la manière propre aux Juifs d’exprimer leurs sensations, manière que nous avons déjà caractérisée. Pour un musicien juif, il ne peut donc exister qu’une seule source d’art populaire juif : celui qui a cours dans les synagogues et qui a pour thème le culte de Jéhovah. Nous sommes tout disposé à admirer en toute sincérité la noblesse des cultes religieux en leur pureté originelle, mais pour autant nous nous voyons forcé de convenir que le culte juif s’est transmis d’une façon désastreuse et fondamentalement altérée. Comme dans toute la vie du peuple juif, nous remarquons l’absence d’une vie intérieure, créatrice de renouveau ; tout est resté figé, morne, tant dans la forme que dans le fond. Exprimer une chose qui a perdu toute vitalité, qui n’est plus qu’une réalité caduque, cela correspond à un anéantissement et perd toute signification. Pour s’en convaincre, il suffit de se rendre dans une synagogue et l’on sera frappé par le grotesque que nous révèle le chant religieux. On ne sait ce qui l’emporte en nous de la répugnance, de l’horreur, ou du ridicule, lorsque nous entendons les gargouillements, les hurlements et les bourdonnements qui s’y confondent. Aucune caricature, si méchante fût-elle, ne saurait donner une impression plus repoussante du chant juif, que ce que nous voyons, en leur naïve nudité. On a tenté, il est vrai, dans les sphères supérieures une réforme et on a voulu restaurer le chant juif dans sa pureté traditionnelle ; mais les efforts conscients et voulus de l’intelligence de certains Juifs cultivés ne purent rien tenter contre l’habitude séculaire. Ce fut là aussi un essai de haut en bas et par cela même, il fut condamné à l’impuissance finale. Or, dans ces conditions, toute tentative essayée par le Juif cultivé pour se retremper dans l’âme de son peuple afin de satisfaire à son besoin artistique, devenait vaine et il ne pouvait dans le peuple trouver le miroir de son intelligence. À la recherche du spontané, il ne trouve devant lui qu’une déformation enlaidie du réfléchi qui est son essence même.}
  \chunks{Ist dieses Zurückgehen auf den Volksquell bei dem gebildeten Juden, wie bei jedem Künstler überhaupt, ein absichtsloses, durch die Natur der Sache mit unbewußter Nothwendigkeit gebotenes, so trägt sich auch der hier empfangene Eindruck ebenso unbeabsichtigt, und daher mit unüberwindlicher Beherrschung seiner ganzen Anschauungsweise, auf seine Kunstproductionen über. Jene Melismen und Rhythmen des Synagogengesanges nehmen seine musikalische Phantasie ganz in der Weise ein, wie das unwillkürliche Innehaben der Weisen und Rhythmen unsres Volksliedes und Volkstanzes die eigentliche gestaltende Kraft der Schöpfer unsrer Kunstgesang- und Instrumental-Musik ausmachte. Dem musikalischen Wahrnehmungsvermögen des gebildeten Juden ist daher aus dem weiten Kreise des Volksthümlichen wie Künstlerischen in unsrer Musik nur Das erfaßbar, was ihn überhaupt als verständlich anmuthet: verständlich, und zwar so verständlich, daß er es künstlerisch zu verwenden vermöchte, ist ihm aber nur Dasjenige, was durch irgend eine Annäherung jener jüdisch-musikalischen Eigenthümlichkeit ähnelt. Würde der Jude bei seinem Hinhorchen auf unser naives, wie bewußt gestaltendes musikalisches Kunstwesen, das Herz und den Lebensnerven desselben zu ergründen sich bemühen, so müßte er aber inne werden, daß seiner musikalischen Natur hier in Wahrheit nicht das Mindeste ähnelt, und das gänzlich Fremdartige dieser Erscheinung müßte ihn dermaßen zurückschrecken, daß er unmöglich den Muth zur Mitwirkung bei unsrem Kunstschaffen sich erhalten könnte. Seine ganze Stellung unter uns verführt den Juden jedoch nicht zu so innigem Eindringen in unser Wesen: entweder mit Absicht (sobald er seine Stellung zu uns erkennt,) oder unwillkürlich (sobald er uns überhaupt gar nicht verstehen kann,) horcht er daher auf unser Kunstwesen und dessen lebengebenden inneren Organismus nur ganz oberflächlich hin, und vermöge dieses theilnahmlosen Hinhorchens allein können sich ihm äußerliche Aehnlichkeiten mit dem seiner Anschauung einzig Verständlichen, seinem besonderen Wesen Eigenthümlichen, darstellen. Ihm wird daher die zufälligste Aeußerlichkeit der Erscheinungen auf unsrem musikalischen Lebens- und Kunstgebiete als deren Wesen gelten müssen, daher seine Empfängnisse davon, wenn er sie als Künstler uns zurückspiegelt, uns fremdartig, kalt, sonderlich, gleichgiltig, unnatürlich und verdreht erscheinen, so daß jüdische Musikwerke auf uns oft den Eindruck hervorbringen, als ob z. B. ein Goethesches Gedicht im jüdischen Jargon uns vorgetragen würde.}
         {Si pour le Juif cultivé, comme pour tout autre, le retour à l’âme populaire est quelque chose d’involontaire et d’inconscient, il est évident que l’impression qu’il en retire est tout aussi inconsciente et que celle-ci se reportera impérieusement sur ses productions artistiques et que ses créations s’en ressentiront puissamment. De même que nos chansons et nos danses populaires ont une action sur les créateurs de notre art musical et instrumental, de même les mélismes et les rythmes des chants de la synagogue s’emparent de l’imagination musicale du Juif cultivé. Il ne peut donc comprendre dans notre production musicale, tant populaire qu’artistique, que ce qui a un trait commun ou quelque analogie avec la musique juive, et cela seul, il pourrait l’utiliser pour ses créations d’art. Certes, si le Juif avait conscience qu’il n’existe rien dans notre art, populaire ou savant, qui soit de la même essence que celui de sa race, s’il en sondait l’âme et s’il prenait conscience qu’il est pour nous un étranger, il reculerait épouvanté et ne tenterait pas une collaboration déconcertante. Mais le Juif ne peut nous connaître, volontairement ou non, il ne s’arrête qu’aux faits extérieurs et superficiels. Et trouvant dans l’apparence fortuite de certaines œuvres, une analogie plus ou moins existante, il consacre celle-ci comme si elle était l’essence même de nos créations. Aussi, lorsqu’il lui arrive de vouloir nous faire part de ses impressions artistiques, celles-ci nous paraissent froides, lointaines, indifférentes, contre-nature et déformées. Nous avons la même impression en écoutant de la musique juive que si nous entendions une poésie de Goethe récitée en jargon juif.}
  \chunks{Wie in diesem Jargon mit wunderlicher Ausdruckslosigkeit Worte und Constructionen durcheinandergeworfen werden, so wirft der jüdische Musiker auch die verschiedenen Formen und Stylarten aller Meister und Zeiten durch einander. Dicht neben einander treffen wir da im buntesten Chaos die formellen Eigenthümlichkeiten aller Schulen angehäuft. Da es sich bei diesen Productionen immer nur darum handelt, daß überhaupt geredet werden soll, nicht aber um den Gegenstand, welcher sich des Redens erst verlohnte, so kann dieses Geplapper eben auch nur dadurch irgendwie für das Gehör anregend gemacht werden, daß es durch den Wechsel der äußerlichen Ausdrucksweise jeden Augenblick eine neue Reizung zur Aufmerksamkeit darbietet. Die innerliche Erregung, die wahre Leidenschaft findet ihre eigenthümliche Sprache in dem Augenblicke, wo sie, nach Verständniß ringend, zur Mittheilung sich anläßt: der in dieser Beziehung von uns bereits näher charakterisirte Jude hat keine wahre Leidenschaft, am allerwenigsten eine Leidenschaft, welche ihn zum Kunstschaffen aus sich drängte. Wo diese Leidenschaft nicht vorhanden ist, da ist aber auch keine Ruhe anzutreffen: wahre, edle Ruhe ist nichts Anderes, als die durch Resignation beschwichtigte Leidenschaft. Wo der Ruhe nicht die Leidenschaft vorangegangen ist, erkennen wir nur Trägheit: der Gegensatz der Trägheit ist aber nur jene prickelnde Unruhe, die wir in jüdischen Musikwerken von Anfang bis zu Ende wahrnehmen, außer da, wo sie jener geist- und empfindungslosen Trägheit Platz macht. Was so der Vornahme der Juden, Kunst zu machen, entsprießt, muß daher nothwendig die Eigenschaft der Kälte, der Gleichgiltigkeit, bis zur Trivialität und Lächerlichkeit an sich haben, und wir müssen die Periode des Judenthums in der modernen Musik geschichtlich als die der vollendeten Unproductivität, der verkommenden Stabilität bezeichnen.}
         {De même que dans ce jargon on observe une énorme indigence d’expressions à laquelle on supplée par un tohu-bohu de constructions bizarres et jetées pêle-mêle, de même le musicien juif mêle toutes les formes et tous les styles de tous les maîtres et de toute époque. Les particularités spéciales à chaque école y ont leur place dans un chaos extrêmement confus. L’essentiel de ces compositions résidant non dans le fond, mais dans la forme, cela devient un bavardage insipide qui, pour provoquer l’attention, doit avoir nécessairement recours à des moyens d’expression tout à fait extérieurs. L’émotion intérieure, la passion vraie trouvera sa langue propre, au moment même où luttant pour se faire comprendre, elle se communique à nous. Le Juif dont nous avons parlé n’éprouvant aucune passion véritable, ne saurait donc ressentir le besoin de création artistique. Où cette passion fait défaut, il n’est point de sérénité possible : celle-ci n’étant autre chose que la passion tempérée par la résignation. On ne trouve qu’inertie, là où la passion n’a pas précédé le calme. Cette inertie est loin d’exister dans les productions musicales juives dans lesquelles règne au contraire une fébrilité excessive, exception faite des endroits où elle fait place à un abandon complet d’idées et de sentiments. Ce qui caractérisera donc le mieux les créations artistiques juives et cela jusqu’à la trivialité et au ridicule, sera un cachet de froideur et d’insensibilité ; par conséquent, la période historique de la musique juive de notre société peut être considérée comme celle de la stérilité complète et du déséquilibre.}
% Exemple de Mendelssohn
  \chunks{An welcher Erscheinung wird uns dies Alles klarer, ja an welcher konnten wir es einzig fast inne werden, als an den Werken eines Musikers jüdischer Abkunft, der von der Natur mit einer specifisch musikalischen Begabung ausgestattet war, wie wenige Musiker überhaupt vor ihm? Alles, was sich bei der Erforschung unsrer Antipathie gegen jüdisches Wesen der Betrachtung darbot, aller Widerspruch dieses Wesens in sich selbst und uns gegenüber, alle Unfähigkeit desselben, außerhalb unsres Bodens stehend, dennoch auf diesem Boden mit uns verkehren, ja sogar die ihm entsprossenen Erscheinungen weiter entwickeln zu wollen, steigern sich zu einem völlig tragischen Conflict in der Natur, dem Leben und Kunstwirken des frühe verschiedenen Felix Mendelssohn-Bartholdy. Dieser hat uns gezeigt, daß ein Jude von reichster specifischer Talentfülle sein, die feinste und mannigfaltigste Bildung, das gesteigertste, zartestempfindende Ehrgefühl besitzen kann, ohne durch die Hilfe aller dieser Vorzüge es je ermöglichen zu können, auch nur ein einziges Mal die tiefe, Herz und Seele ergreifende Wirkung auf uns hervorzubringen, welche wir von der Kunst erwarten, weil wir sie dessen fähig wissen, weil wir diese Wirkung zahllos oft empfunden haben, sobald ein Heros unsrer Kunst, so zu sagen, nur den Mund aufthat, um zu uns zu sprechen. Kritikern von Fach, welche hierüber zu gleichem Bewußtsein mit uns gelangt sein sollten, möge es überlassen sein, diese zweifellos gewisse Erscheinung aus den Einzelnheiten der Mendelssohnschen Kunstproductionen nachweislich zu bestätigen: uns genüge es hier, zur Verdeutlichung unsrer allgemeinen Empfindung uns zu vergegenwärtigen, daß bei Anhörung eines Tonstückes dieses Componisten wir uns nur dann gefesselt fühlen konnten, wenn nichts Anderes unsre, mehr oder weniger nur unterhaltungssüchtigen Phantasie, als Vorführung, Reihung und Verschlingung der feinsten, glättesten und kunstfertigsten Figuren, wie im wechselnden Farben- und Formenreize des Kaleidoskopes, vorgeführt wurden, – nie aber da, wo diese Figuren die Gestalt tiefer und markiger menschlicher Herzensempfindungen anzunehmen bestimmt waren[2]. Für diesen letzteren Fall hörte für Mendelssohn selbst alles formelle Productionsvermögen auf, weßhalb er denn namentlich da, wo er sich, wie im Oratorium, zum Drama anläßt, ganz offen nach jeder formellen Einzelnheit, welche diesem oder jenem zum Stylmuster gewählten Vorgänger als individuell charakteristisches Merkmal besonders zu eigen war, greifen mußte. Bei diesem Verfahren ist es noch bezeichnend, daß der Componist für seine ausdrucksunfähige moderne Sprache besonders unsren alten Meister Bach als nachzuahmendes Vorbild sich erwählte. Bachs musikalische Sprache bildete sich in der Periode unsrer Musikgeschichte, in welcher die allgemeine musikalische Sprache eben noch nach der Fähigkeit individuelleren, sicheren Ausdruckes rang: das rein Formelle, Pedantische haftete noch so stark an ihr, daß ihr rein menschlicher Ausdruck bei Bach, durch die ungeheure Kraft seines Genies, eben erst zum Durchbruche kam. Die Sprache Bachs steht zur Sprache Mozarts und endlich Beethovens in dem Verhältnisse, wie die ägyptische Sphinx zur griechischen Menschenstatue: wie die Sphinx mit dem menschlichen Gesichte aus dem Thierleibe erst noch herausstrebt, so strebt Bachs edler Menschenkopf aus der Perücke hervor. Es liegt eine unbegreiflich gedankenlose Verwirrung des luxuriösen Musikgeschmackes unsrer Zeit darin, daß wir die Sprache Bachs neben derjenigen Beethovens ganz zu gleicher Zeit uns vorsprechen lassen, und uns weismachen können, in den Sprachen Beider läge nur ein individuell formeller, keinesweges aber ein culturgeschichtlich wirklicher Unterschied vor. Der Grund hiervon ist aber leicht einzusehen: die Sprache Beethovens kann nur von einem vollkommenen, ganzen, warmen Menschen gesprochen werden, weil sie eben die Sprache eines so vollendeten Musikmenschen war, daß dieser mit nothwendigem Drange über die absolute Musik hinaus, deren Bereich er bis an seine äußersten Grenzen ermessen und erfüllt hatte, uns den Weg der Befruchtung aller Künste durch die Musik als ihre einzige erfolgreiche Erweiterung angewiesen hat. Die Sprache Bachs hingegen kann füglich von einem sehr fertigen Musiker, wenn auch nicht im Sinne Bachs, nachgesprochen werden, weil das Formelle in ihr noch das Ueberwiegende, und der rein menschliche Ausdruck noch nicht das so bestimmt Vorherrschende ist, daß in ihr bereits unbedingt nur das Was ausgesagt werden könnte oder müßte, da sie eben noch in der Gestaltung des Wie begriffen ist. Die Zerflossenheit und Willkürlichkeit unsres musikalischen Styles ist durch Mendelssohns Bemühen, einen unklaren fast nichtigen Inhalt so interessant und geistblendend wie möglich auszusprechen, wenn nicht herbeigeführt, so doch auf die höchste Spitze gesteigert worden. Rang der Letzte in der Kette unsrer wahrhaften Musikheroen, Beethoven, mit höchstem Verlangen und wunderwirkendem Vermögen nach klarstem, sicherstem Ausdrucke eines unsäglichen Inhaltes durch scharfgeschnittene plastische Gestaltung seiner Tonbilder, so verwischt dagegen Mendelssohn in seinen Productionen diese gewonnenen Gestalten zum zerfließenden, phantastischen Schattenbilde, bei dessen unbestimmtem Farbenschimmer unsre launenhafte Einbildungskraft willkürlich angeregt, unser rein menschliches inneres Sehnen nach deutlichem künstlerischem Schauen aber kaum nur mit der Hoffnung auf Erfüllung berührt wird. Nur da, wo das drückende Gefühl von dieser Unfähigkeit sich der Stimmung des Componisten zu bemächtigen scheint, und ihn zu dem Ausdrucke weicher und schwermüthiger Resignation hindrängt, vermag sich uns Mendelssohn charakteristisch darzustellen, charakteristisch in dem subjectiven Sinne einer zartsinnigen Individualität, die sich der Unmöglichkeit gegenüber ihre Ohnmacht eingesteht. Dies ist, wie wir sagten, der tragische Zug in Mendelssohns Erscheinung; und wenn wir auf dem Gebiete der Kunst an die reine Persönlichkeit unsre Theilnahme verschenken wollten, so dürften wir sie Mendelssohn in starkem Maße nicht versagen, selbst wenn die Kraft dieser Theilnahme durch die Beachtung geschwächt würde, daß das Tragische seiner Situation Mendelssohn mehr anhing, als es ihm zum wirklichen, schmerzlichen und läuternden Bewußtsein kam.}
         {À quelles productions le verrons-nous plus nettement que dans celles d’un compositeur d’origine juive, que la Nature avait pourtant pourvu de grandes qualités musicales, comme peu de musiciens en avaient eu avant lui ? Tout ce qui a pu alimenter les recherches de notre antipathie pour la nature juive, tout ce que cette nature présente de contradictions avec nous, l’incapacité dans laquelle elle se trouve, n’étant pas de notre sol, à vouloir se mêler avec nous sur ce sol et, enfin, l’impossibilité qui existe pour elle à vouloir même cultiver ses éléments propres, voilà quelles furent les origines du tragique conflit qui mit aux prises la nature et les goûts artistiques d’un musicien mort prématurément : Félix Mendelssohn-Bartholdy. Celui-ci nous a démontré qu’un Juif, si talentueux soit-il, aurait-il la culture la plus parfaite et la plus délicate, nourrirait-il l’ambition la plus élevée et la plus légitime, ne parviendra néanmoins jamais à produire sur notre cœur et sur notre âme l’impression splendide que nous sommes en droit d’attendre de l’art et qui nous est révélée par ailleurs dès qu’un des nôtres, dès qu’un héros de notre art, descelle ses lèvres pour nous parler. Nous laisserons à des critiques de métier qui sont imbus de la même pensée que nous, le soin de nous expliquer par des preuves tirées de l’œuvre musicale de Mendelssohn le bien-fondé de notre assertion. Il nous suffira pour le moment de rappeler qu’il nous était impossible de nous sentir émus à l’audition d’une œuvre de ce musicien puisque nous cherchions, non pas à faire défiler devant nous comme un kaléidoscope aux formes les plus variées et aux couleurs les plus vives, mais bien l’expression de sentiments parlant à notre cœur et touchant aux fibres mêmes de notre intimité et de notre humanité. Dans ce cas précis, toute faculté créatrice faisait défaut à Mendelssohn ; aussi s’était-il vu obligé, lorsque dans l’oratorio il touchait au drame, de calquer toutes les particularités essentiellement propres à ses prédécesseurs et de prendre ceux-ci comme modèles à suivre. Et ce qui est encore plus remarquable, c’est que ce compositeur fixait de préférence son choix, pour son langage sans expression personnelle, sur notre vieux maître Bach. Or, il ne faut pas oublier que Bach forma son génie et sa langue en une période où la langue musicale était encore à la recherche d’une expression plus individuelle et plus positive. Empêtrée dans le formalisme et le pédantisme, c’est grâce à Bach que la langue musicale doit de trouver pour la première fois une expression spécifiquement humaine. Mais la langue de Bach est à celle de Mozart et de Beethoven, ce que le sphinx égyptien est à la statue grecque. De même que le sphinx égyptien tend à se défaire de sa forme animale, de même la noble figure de Bach s’efforce de se dépouiller de sa perruque. Pour soutenir qu’il n’existe d’autre différence entre la langue de Bach et celle de Beethoven que l’individualité de style et de pensée propre à chacun de ces deux maîtres, il faut toute l’incohérence et la futilité de nos contemporains. En réalité, c’est au degré de civilisation qu’est due cette différence. Elle est d’ailleurs facile à comprendre. La langue de Beethoven ne saurait être parlée que par un homme complet, puissant et chaleureux, puisqu’elle était précisément la langue d’un musicien si génial, que dépassant d’un coup le domaine de la musique absolue, il nous indiquait la voie de la fécondation de tous les arts par la musique comme étant le royaume même où elle devait régner en maîtresse absolue. La langue de Bach, au contraire est encore trop formaliste, trop rigoureusement étroite, pour qu’elle ne puisse donner lieu à des imitations faites par un musicien adroit. L’expression individuelle n’y prédomine pas assez pour que le fonds seul nous attire, alors qu’en réalité elle est encore à se demander comment elle peut s’exprimer. Il est évident que si ce n’est pas à Mendelssohn lui-même que nous devons, de par ses efforts à exprimer par des artifices de métier éblouissants et étonnants, des choses vagues et insignifiantes, la déliquescence de notre style musical actuel, il y a néanmoins grandement contribué et a atteint le plus haut degré dans cette façon d’exprimer ses sentiments. Beethoven, qui est le dernier en date de nos héros de la musique, s’était efforcé avec une volonté ardente et une puissance admirable, à créer des formes musicales bien nettes, bien plastiques, représentatives de ses tableaux de pure éloquence ; Mendelssohn, au contraire, fait de ces images aux contours bien définis quelque chose de vague, de fantaisiste, plein d’ombres. Notre imagination peut être excitée par leur arbitraire capricieux, mais notre désir de nous trouver en face de sentiments humains, clairement exprimés, reste forcément inassouvi. Et nous ne rencontrons l’artiste vrai que dans les passages où il a le sentiment de son incapacité et où son âme défaillante laisse parler sa résignation, faite d’une certaine noblesse et de mélancolie. Alors, Mendelssohn se représente subjectivement, mais son individualité fine et tendre est obligée de manifester son impuissance et sa faiblesse. Et c’est là, comme nous l’avons dit, que réside le côté tragiquement cruel du cas de ce compositeur ; et c’est pourquoi, si nous devons accorder dans le domaine de l’art quelque sympathie à l’individu pris en lui-même, nous ne pourrions à plus forte raison la refuser à Mendelssohn, bien que nous restions persuadés que le tragique de son cas était inhérent au compositeur, celui-ci n’en ayant pas réellement conscience.}
  \chunks{Eine ähnliche Theilnahme vermag aber kein anderer jüdischer Componist uns zu erwecken. Ein weit und breit berühmter jüdischer Tonsetzer unsrer Tage hat sich mit seinen Productionen einem Theile unsrer Oeffentlichkeit zugewendet, in welchem die Verwirrung alles musikalischen Geschmackes von ihm weniger erst zu veranstalten, als nur noch auszubeuten war. Das Publicum unsrer heutigen Operntheater ist seit längerer Zeit nach und nach gänzlich von den Anforderungen abgebracht worden, welche nicht etwa an das dramatische Kunstwerk selbst, sondern überhaupt an Werke des guten Geschmackes zu stellen sind. Die Räume dieser Unterhaltungslocale füllen sich meistens nur mit jenem Theile unsrer bürgerlichen Gesellschaft, bei welchem der einzige Grund zur wechselnden Vornahme irgend welcher Beschäftigung die Langeweile ist: die Krankheit der Langeweile ist aber nicht durch Kunstgenüsse zu heilen, denn sie kann absichtlich gar nicht zerstreut, sondern nur durch eine andere Form der Langeweile über sich selbst getäuscht werden. Die Besorgung dieser Täuschung hat nun jener berühmte Operncomponist zu seiner künstlerischen Lebensaufgabe gemacht. Es ist zwecklos, den Aufwand künstlerischer Mittel näher zu bezeichnen, deren er sich zur Erreichung seiner Lebensaufgabe bediente: genug, daß er es, wie wir aus dem Erfolge ersehen, vollkommen verstand, zu täuschen, und dieses namentlich damit, daß er jenen von uns näher charakterisirten Jargon seiner gelangweilten Zuhörerschaft[3] als modern pikante Aussprache aller der Trivialitäten aufheftete, welche ihr so widerholt oft schon in ihrer natürlichen Albernheit vorgeführt worden waren. Daß dieser Componist auch auf Erschütterungen und auf die Benutzung der Wirkung von eingewobenen Gefühlskatastrophen bedacht war, darf Niemanden befremden, der da weiß, wie nothwendig dergleichen von Gelangweilten gewünscht wird; daß hierin ihm seine Absicht aber auch gelingt, darf denjenigen nicht wundern, der die Gründe bedenkt, aus denen unter solchen Umständen ihm Alles gelingen muß. Dieser täuschende Componist geht sogar so weit, daß er sich selbst täuscht, und dieses vielleicht ebenso absichtlich, als er seine Gelangweilten täuscht. Wir glauben wirklich, daß er Kunstwerke schaffen möchte, und zugleich weiß, daß er sie nicht schaffen kann: um sich aus diesem peinlichen Conflicte zwischen Wollen und Können zu ziehen, schreibt er für Paris Opern, und läßt diese dann leicht in der übrigen Welt aufführen, – heut' zu Tage das sicherste Mittel, ohne Künstler zu sein, doch Kunstruhm sich zu verschaffen. Unter dem Drucke dieser Selbsttäuschung, welche nicht so mühelos sein mag, als man denken könnte, erscheint er uns fast gleichfalls in einem tragischen Lichte: das rein Persönliche in dem gekränkten Interesse macht die Erscheinung aber zu einer tragikomischen, wie überhaupt das Kaltlassende, wirklich Lächerliche, das Bezeichnende des Judenthumes für diejenige Kundgebung desselben ist, in welcher der berühmte Componist sich uns in Bezug auf die Musik zeigt.}
         {Pourtant, Mendelssohn reste le seul compositeur Juif, éveillant en nous une pareille sympathie. Il s’est trouvé un autre compositeur Juif contemporain, et qui est universellement connu, mais il n’a en vue que de créer ses productions musicales afin d’exploiter — puisqu’on ne pouvait plus le corrompre — le goût public. Ceux qui vont dans nos théâtres modernes d’opéra ont depuis longtemps pris pour habitude d’être très peu sévères à l’égard de toute œuvre dramatique ; il en est d’ailleurs ainsi pour la généralité des productions de bon goût. Les salles de spectacle sont, dans la plupart des cas, remplies par des gens appartenant à notre bourgeoisie et qui à la base de toutes leurs actions ont comme mobile : l’ennui, Mais l’ennui ne saurait être guéri par une joie artistique. L’ennui ne s’en va pas : on n’arrive qu’à faire naître l’illusion en variant à l’infini les formes mêmes de l’ennui. C’est cette illusion qu’a cherché à produire ce célèbre compositeur d’opéras et ce fut là sa mission artistique. Pas n’est besoin de spécifier l’emploi des moyens auxquels il s’est astreint pour atteindre son but. Il lui a suffi, et son succès en fait foi, à créer l’illusion et il a même réussi à imposer à ses auditeurs le jargon juif que nous avons déjà caractérisé plus haut[2]. Ce compositeur a recherché et employé dans ses œuvres les effets dus aux conflits sentimentaux et rien n’est plus naturel, car il est notoire que les gens qui s’ennuient se distraient volontiers aux sensations de cet ordre. Il n’y a pas à s’étonner que la chose lui ait réussi, quand on considère les raisons pour lesquelles, en de pareilles circonstances, tout doit lui réussir. Ce qu’il y a de plus curieux, c’est que ce compositeur illusionniste s’illusionne certainement lui-même, peut-être avec intention, ainsi qu’il le fait pour son public de blasés. Nous croyons réellement qu’il voudrait créer des œuvres d’art, mais qu’il a conscience de son impuissance. Pour sortir de ce dilemne entre vouloir et pouvoir, il écrit des opéras à l’intention de Paris, et les fait alors représenter dans le monde entier. C’est le plus sûr moyen de se faire passer à présent pour un artiste, sans avoir le moins du monde le sens artistique. Cette auto-illusion qui doit certainement être plus pénible qu’on ne le suppose, nous le ferait également apparaître sous un côté tragique ; mais le rôle joué par l’intérêt au point de vue personnel fait plutôt de ce personnage quelque chose de tragi-comique. L’impression de froideur et de profond ridicule qu’il nous produit révèle en somme les caractères distinctes du judaïsme dans la musique.}
  \chunks{Aus der genaueren Betrachtung der vorgeführten Erscheinungen, welche wir durch die Ergründung und Rechtfertigung unsres unüberwindlichen Widerwillens gegen jüdisches Wesen verstehen lernen konnten, ergiebt sich uns besonders nun die dargethane Unfähigkeit unsrer musikalischen Kunstepoche. Hätten die näher erwähnten beiden jüdischen Componisten[4] in Wahrheit unsre Musik zu höherer Blüthe gefördert, so müßten wir uns nur eingestehen, daß unser Zurückbleiben hinter ihnen auf einer bei uns eingetretenen organischen Unfähigkeit beruhe: dem ist aber nicht so; im Gegentheile stellt sich das individuelle rein musikalische Vermögen gegen vergangene Kunstepochen als eher vermehrt denn vermindert heraus. Die Unfähigkeit liegt in dem Geiste unsrer Kunst selbst, welche nach einem anderen Leben verlangt, als das künstliche es ist, das ihr mühsam jetzt erhalten wird. Die Unfähigkeit der musikalischen Kunstart selbst wird uns in Mendelssohns, des specifisch ungemein begabten Musikers, Kunstwirken dargethan; die Nichtigkeit unsrer ganzen Oeffentlichkeit, ihr durchaus unkünstlerisches Wesen und Verlangen, wird uns aber aus den Erfolgen jenes berühmten jüdischen Operncomponisten auf das Ersichtlichste klar. Dies sind die wichtigen Punkte, die jetzt die Aufmerksamkeit eines Jeden, welcher es ehrlich mit der Kunst meint, ausschließlich auf sich zu ziehen haben: hierüber haben wir zu forschen, uns zu fragen und zum deutlichen Verständniß zu bringen. Wer diese Mühe scheut, wer sich von dieser Erforschung abwendet, entweder weil ihn kein Bedürfniß dazu treibt, oder weil er die mögliche Erkenntniß von sich abweist, die ihn aus dem trägen Geleise eines gedanken- und gefühllosen Schlendrians heraustreiben müßte, den eben begreifen wir jetzt mit unter der Kategorie der ‚Judenschaft in der Musik‘. Dieser Kunst konnten sich die Juden nicht eher bemächtigen, als bis in ihr das darzuthun war, was sie in ihr erweislich eben offengelegt haben: ihre innere Lebensunfähigkeit. So lange die musikalische Sonderkunst ein wirkliches organisches Lebensbedürfniß in sich hatte, bis auf die Zeiten Mozarts und Beethovens, fand sich nirgends ein jüdischer Componist: unmöglich konnte ein diesem Lebensorganismus gänzlich fremdes Element an den Bildungen dieses Lebens theilnehmen. Erst wenn der innere Tod eines Körpers offenbar ist, gewinnen die außerhalb liegenden Elemente die Kraft, sich seiner zu bemächtigen, aber nur um ihn zu zersetzen; dann löst sich wohl das Fleisch dieses Körpers in wimmelnde Viellebigkeit von Würmern auf: wer möchte aber bei ihrem Anblicke den Körper selbst noch für lebendig halten? Der Geist, das ist: das Leben, floh von diesem Körper hinweg zu wiederum Verwandtem, und dieses ist nur das Leben selbst: nur im wirklichen Leben können auch wir den Geist der Kunst wiederfinden, nicht bei ihrer Würmer-zerfressenen Leiche. –}
         {En examinant soigneusement les faits que nous avons articulés et qui nous ont été connus par la recherche de notre antipathie pour l’élément judaïque, il ressort clairement ce que nous appellerons l’incapacité musicale de notre époque. Si les deux compositeurs juifs[3] avaient porté notre musique à un plus haut degré d’épanouissement, nous devrions en toute sincérité reconnaître notre retard sur eux et l’attribuer à notre nature même. Mais ce n’est pas le cas. Bien au contraire. Notre faculté individuelle artistique, comparée aux temps passés, semble s’être avivée et être agrandie. La stérilité réside bien plus dans notre art lui-même et dans la vie artificielle que nous lui conservons. Elle éclate plus particulièrement dans l’effort artistique d’un musicien pourtant extrêmement doué ; Mendelssohn. Par contre, le peu de valeur de notre public, son manque général de culture, ses goûts anti-artistiques sont surabondamment prouvés par le succès obtenu par le célèbre compositeur juif d’opéras dont nous avons parlé. C’est sur ces deux points principaux que doit se concentrer l’attention de ceux qui aiment notre art et c’est là-dessus que doit porter notre enquête afin d’en arriver à une conception précise. Quant à ceux qu’effraie cette recherche ou qui s’en désintéressent, ceux qui préfèrent marcher dans la vieille ornière de la routine, ils méritent d’être catalogués dans cette catégorie spéciale que nous avons appelée le « Judaïsme dans la musique ». Il était impossible aux Juifs de s’emparer de cet art avant que celui-ci fût devenu une chose sans vie ; aussi longtemps que la musique possédait en soi une vie organique intense, c’est-à-dire jusqu’à Mozart et Beethoven, nous ne trouvions pas trace de Juifs dans la musique. Ce n’est que lorsque la vie organique de celle-ci périclita, que les éléments extérieurs prirent suffisamment d’empire pour s’en rendre maîtres et la décomposer. La chair d’un tel cadavre, grouillant de vers, peut se dissoudre, mais il n’est personne qui puisse pour autant le considérer comme une chose vivante. La vie a fui ce corps pour aller retrouver la vie ; ce n’est que dans la vie même que nous retrouverons l’esprit artistique qui doit nous inspirer et non sur ce cadavre décomposé et dévoré par les vers.}
  \chunks{Ich sagte oben, die Juden hätten keinen wahren Dichter hervorgebracht. Wir müssen nun hier Heinrich Heine’s erwähnen.}
         {J’ai dit plus haut que les Juifs n’avaient pas produit de véritable poète. Il est temps à présent de parler de Henri Heine.}
  \chunks{Zur Zeit, da Goethe und Schiller bei uns dichteten, wissen wir allerdings von keinem dichtenden Juden: zu der Zeit aber, wo das Dichten bei uns zur Lüge wurde, unsrem gänzlich unpoetischen Lebenselemente alles Mögliche, nur kein wahrer Dichter mehr entsprießen wollte, da war es das Amt eines sehr begabten dichterischen Juden, diese Lüge, diese bodenlose Nüchternheit und jesuitische Heuchelei unsrer immer noch poetisch sich gebaren wollenden Dichterei mit hinreißendem Spotte aufzudecken. Auch seine berühmten musikalischen Stammesgenossen geißelte er unbarmherzig für ihr Vorgeben, Künstler sein zu wollen; keine Täuschung hielt bei ihm vor: von dem unerbittlichen Dämon des Verneinens Dessen, was verneinenswerth schien, ward er rastlos vorwärtsgejagt, durch alle Illusionen moderner Selbstbelügung hindurch, bis auf den Punkt, wo er nun selbst wieder sich zum Dichter log, und dafür auch seine gedichteten Lügen von unsren Componisten in Musik gesetzt erhielt. – Er war das Gewissen des Judenthumes, wie das Judenthum das üble Gewissen unsrer modernen Civilisation ist.}
         {À l’époque de Goethe et de Schiller, on ne connut pas de poète juif. Ce n’est que lorsque la poésie devint, chez nous, quelque chose de mensonger et d’hypocrite, et qu’elle ne fut plus capable de produire un véritable poète, que parut alors un Juif très doué pour la poésie et qui prit pour tâche de railler d’une façon cinglante notre indigence et notre hypocrisie jésuitique. De même, il flagella tout aussi impitoyablement ses coreligionnaires musiciens qui se prétendaient des artistes. Aucune illusion ne tint devant lui. Le démon de la négation le poussa sans trêve ; il renia tout ce qui lui parut bon à renier, mais se mentit à lui-même en se croyant un poète et reçut comme châtiment ses poésies rythmées par nos compositeurs. Heine fut la conscience du judaïsme, de même que le judaïsme est la néfaste conscience de notre civilisation moderne.}
  \chunks{Noch einen Juden haben wir zu nennen, der unter uns als Schriftsteller auftrat. Aus seiner Sonderstellung als Jude trat er Erlösung suchend unter uns: er fand sie nicht und mußte sich bewußt werden, daß er sie nur mit auch unsrer Erlösung zu wahrhaften Menschen finden können würde. Gemeinschaftlich mit uns Mensch werden, heißt für den Juden aber zu allernächst so viel als: aufhören, Jude zu sein. Börne hatte dies erfüllt. Aber gerade Börne lehrt auch, wie diese Erlösung nicht in Behagen und gleichgiltig kalter Bequemlichkeit erreicht werden kann, sondern daß sie, wie uns, Schweiß, Noth, Aengste und Fülle des Leidens und Schmerzes kostet. Nehmt rücksichtslos an diesem, durch Selbstvernichtung wiedergebärenden Erlösungswerke theil, so sind wir einig und ununterschieden! Aber bedenkt, daß nur Eines eure Erlösung von dem auf euch lastenden Fluche sein kann: die Erlösung Ahasvers, – der Untergang!\ornamentbreak}
         {Nous devons encore parler d’un autre Juif qui s’est fait connaître chez nous comme écrivain. Il quitta sa position spéciale de Juif et chercha auprès de nous la Rédemption. Il ne la trouva pas, mais dut reconnaître qu’il ne saurait la trouver que le jour où, nous aussi, devenus de véritables hommes, nous serions sauvés. Mais devenir homme, correspond pour le Juif à ne plus être Juif. C’est ce que tenta Bœrne. Son exemple prouve qu’on ne saurait trouver la Rédemption dans la quiétude, dans le bien-être ou l’indifférence. Pour l’atteindre, il faut, au contraire, peiner ; elle nous coûte sueurs, tourments, angoisses, misère et douleurs. Prenez part, en toute loyauté, à cette œuvre rédemptrice et nous serons alors unis et tous pareils. Mais songez bien qu’une seule chose peut vous conjurer de la malédiction qui pèse sur vous : la rédemption d’Ahasvérus : l’anéantissement.}
% * * *
% traduction Google Translate
  \chunks{Der mit dem Vorstehenden wesentlich unverändert mitgetheilte Aufsatz erschien, wie ich anfangs erwähnte, vor etwas mehr als achtzehn Jahren, und zwar in der „Neuen Zeitschrift für Musik“.}
         {L'essai, reproduit ici quasiment à l'identique, a été publié, comme je l'ai mentionné au début, il y a un peu plus de dix-huit ans, dans la \frquote{Neue Zeitschrift für Musik}.}
  \chunks{Heute noch ist es mir fast unbegreiflich, wie mein nun kürzlich verstorbener Freund Franz Brendel, der Herausgeber jener Zeitschrift, es über sich vermocht hat, die Veröffentlichung dieses Artikels zu wagen: jedenfalls war der so ernstlich gesinnte, nur die Sache in das Auge fassende, durchaus redliche und biedere Mann gar nicht der Meinung gewesen, hiermit etwas Anderes zu thun als eben, der Erörterung einer die Geschichte der Musik betreffenden, sehr beachtenswerthen Frage den unerläßlich gebührenden Raum gestattet zu haben. Dagegen belehrt ihn nun der Erfolg, mit wem er es zu thun hatte. – Leipzig, an dessen Conservatorium für Musik Brendel als Professor angestellt war, hatte in Folge der langjährigen Wirksamkeit des dort mit Recht und nach Verdienst geehrten Mendelssohn die eigentliche musikalische Judentaufe erhalten: wie ein Berichterstatter sich einmal beklagte, waren blonde Musiker dort zu immer größeren Seltenheit geworden, und der sonst durch seine Universität und seinen bedeutenden Buchhandel in allem deutschen Wesen so regsam sich auszeichnende Ort verlernte im Betreff der Musik sogar die natürlichsten Sympathien jedes, sonst deutschen Städten so willig anhaftenden Lokalpatriotismus ; er ward ausschließlich Judenmusikweltstadt. Der Sturm, welcher sich jetzt gegen Brendel erhob, stieg bis zur Bedrohung seiner bürgerlichen Existenz: mit Mühe verdankte er es seiner Festigkeit und ruhig sich bethätigenden Ueberzeugung, daß man ihn in seiner Stellung am Conservatorium belassen mußte.}
         {Aujourd'hui encore, j'ai du mal à comprendre comment mon ami Franz Brendel, récemment décédé et rédacteur en chef de cette revue, a pu oser publier cet article : en tout cas, cet homme sérieux, concentré, d'une honnêteté et d'une droiture irréprochables  n'avait d'autre intention que d'examiner attentivement une question très importante concernant l'histoire de la musique. Or, le succès lui a désormais révélé à qui il avait affaire. – Leipzig, où Brendel enseignait au Conservatoire de musique, avait, grâce à l'œuvre de longue date de Mendelssohn, justement honoré dans la ville, reçu une véritable immersion musicale dans le judaïsme : comme le déplorait un journaliste, les musiciens blonds s'y faisaient de plus en plus rares, et la ville, par ailleurs si active dans tous les domaines allemands grâce à son université et à son important commerce du livre, avait même perdu, en matière de musique, la sympathie la plus naturelle du patriotisme local si facilement attaché aux autres villes allemandes ; elle était devenue une métropole exclusivement juive pour la musique. La tempête qui s'abattit alors sur Brendel devint telle qu'elle menaça sa survie même : ce n'est que par sa fermeté et sa conviction, exprimée avec calme, qu'il put conserver son poste au Conservatoire.}
  \chunks{Was ihm bald zu äußerlicher Ruhe verhalf, war eine sehr charakteristische Wendung, welche die Angelegenheit nach dem ersten unbedachten Aufbrausen des Zornes der Beleidigten nahm.}
         {Ce qui l'a rapidement aidé à retrouver son calme apparent fut un tournant très caractéristique des événements qui suivit l'explosion de colère initiale et irréfléchie de la partie offensée.}
  \chunks{Ich hatte keinesweges im Sinne gehabt, erforderlichen Falles mich als den Verfasser des Aufsatzes zu verläugnen: nur wollte ich verhüten, daß die von mir sehr ernstlich und objectiv aufgefaßte Frage sofort in das rein Persönliche verschleppt würde, was, meiner Meinung nach, alsbald zu erwarten stand, wenn mein Name, also der ’eines jedenfalls auf den Ruhm Anderer neidischen Componisten’, von vornherein in das Spiel gezogen wurde. Deßhalb hatte ich den Artikel mit einem, absichtlich als solchen erkennbaren Pseudonym: K. Freigedank, unterzeichnet. Brendel hatte ich in diesem Betreff meine Absicht mitgetheilt: er war muthig genug, statt, wie dies sofort von befreiender Wirkung für ihn gewesen wäre, den Sturm auf mich hinüberzuleiten, diesen standhaft über sich ergehen zu lassen. Bald erschienen mir Anzeichen dafür, ja deutliche Hinweisungen darauf, daß man mich als den Verfasser erkannt hatte: nie bin ich einer Bezichtigung in diesem Betreff mit einer Abläugnung entgegengetreten. Hiermit erfuhr man genug, um demzufolge die bisher eingehaltene Taktik gänzlich zu verändern. Bisher war jedenfalls nur das gröbere Geschütz des Judenthums gegen den Aufsatz in das Gefecht geführt worden: es zeigte sich kein Versuch, in irgend geistvoller, ja nur geschickter Weise eine Entgegnung zu Stande zu bringen. Gröbliche Anfälle und schimpfende Abwehr der dem Verfasser des Aufsatzes untergelegten, für unsre aufgeklärten Zeiten so schmachvollen, mittelalterlichen Judenhaß-Tendenz, waren das Einzige, was neben absurden Verdrehungen und Fälschungen des Gesagten zum Vorschein kam. Nun aber ward es anders. Jedenfalls nahm sich das höhere Judenthum der Sache an. Das Aergerliche war diesem überhaupt das erregte Aufsehen: sobald man meinen Namen erfuhr, war durch ein Hineinziehen desselben nur noch die Vermehrung dieses Aufsehens zu befürchten. Dieses vermeiden zu können war eben dadurch an die Hand gegeben, daß ich meinem Namen einen Pseudonym substituirt hatte. Es erschien nun räthlich, mich als den Verfasser des Aufsatzes fortan zu ignoriren, und zugleich alles Gerede darüber selbst aufhören zu lassen. Dagegen war ich ja an ganz anderen Seiten anzufassen: ich hatte Kunstschriften veröffentlicht und Opern geschrieben, welch letztere ich doch jedenfalls aufgeführt wissen wollte. Meine systematische Verleumdung und Verfolgung auf diesen Gebieten, mit gänzlichem Secretiren der unangenehmen Judenthumsfrage, versprach jedenfalls die erwünschte Wirkung meiner Bestrafung.}
         {Je n'avais nullement eu l'intention de nier être l'auteur de l'essai si nécessaire ; je souhaitais simplement éviter que la question, que j'examinais avec le plus grand sérieux et la plus grande objectivité, ne soit immédiatement entraînée dans le domaine personnel, ce qui, à mon avis, était inévitable dès lors que mon nom – celui d'un compositeur envieux de la renommée d'autrui – était évoqué. C'est pourquoi j'avais signé l'article d'un pseudonyme, délibérément identifiable comme tel : K. Freigedank. J'avais informé Brendel de mon intention à ce sujet ; au lieu de retourner immédiatement l'attaque contre moi, ce qui aurait été libérateur pour lui, il a eu le courage de la supporter avec constance. J'ai rapidement constaté des signes, voire des indices clairs, que j'avais été reconnu comme l'auteur : je n'ai jamais nié avoir été accusé à ce sujet. Cela a suffi à modifier radicalement la tactique employée jusque-là. Jusqu'alors, seules les attaques les plus grossières contre l'essai avaient été menées, sans aucune tentative de réfutation intelligente, ni même habile. Seules des diatribes vulgaires et des démentis injurieux de l'antisémitisme médiéval attribué à l'auteur – une honte en notre temps éclairé – avaient émergé, accompagnées de distorsions et de falsifications absurdes des propos tenus. Mais la situation avait changé. Au moins, les membres les plus influents de la communauté juive s'étaient emparés de l'affaire. Ce qui l'agaçait particulièrement, c'était l'agitation autour de l'affaire : dès que mon nom était connu, le risque de l'amplifier était bien réel. L'éventualité d'éviter cela m'a été rendue possible par le fait que j'avais substitué un pseudonyme à mon véritable nom. Il semblait désormais judicieux de m'ignorer comme auteur de l'essai et, par la même occasion, de mettre un terme à toute discussion à ce sujet. Cependant, j'étais vulnérable dans des domaines tout à fait différents : j'avais publié des traités sur l'art et composé des opéras, que je souhaitais ardemment voir joués. La diffamation et la persécution systématiques dont j'ai fait l'objet dans ces domaines, conjuguées au secret absolu entourant la question délicate de mon identité juive, promettaient l'effet escompté de ma punition.}
  \chunks{Er wäre gewiß anmaßlich von mir, der ich damals gänzlich zurückgezogen in Zürich lebte, wollte ich eine genauere Bezeichnung des inneren Getriebes der hiermit gegen mich eingeleiteten und in immer weiterer Verbreitung fortgesetzten, umgekehrten Judenverfolgung versuchen. Nur die Erfahrungen, welche Jedermann offenliegen, will ich berichten. Nach der Aufführung des „Lohengrin” in Weimar, im Sommer 1850, traten in der Presse Männer von bedeutendem litterarischen und künstlerischen Rufe, wie Adolf Stahr und Robert Franz, verheißungsvoll hervor, um auf mich und mein Werk das deutsche Publicum aufmerksam zu machen; selbst in Musikblättern von bedenklicher Tendenz tauchten überraschend gewichtige Erklärungen für mich auf. Dies geschah von Seiten jedes der verschiedenen Verfasser aber genau nur einmal. Sofort verstummten sie wieder, und benahmen sich im Verlaufe der Dinge nach Umständen sogar feindselig gegen mich. Dagegen tauchte zunächst ein Freund und Bewunderer des Herrn Ferdinand Hiller, ein Professor Bischoff, in der Kölnischen Zeitung mit der Begründung des von jetzt an gegen mich befolgten Systemes der Verleumdung auf: dieser hielt sich an meine Kunstschriften, und verdrehte meine Idee eines „Kunstwerkes der Zukunft” in die lächerliche Tendenz einer „Zukunftsmusik”, nämlich etwa einer solchen, welche, wenn sie jetzt auch schlecht klänge, mit der Zeit sich doch gut ausnehmen würde. Des Judenthums ward von ihm mit keinem Worte erwähnt, im Gegentheil steifte er sich darauf, Christ und Abkömmling eines Superintendenten zu sein. Dagegen hatte ich Mozart, und selbst Beethoven für Stümper erklärt, wollte die Melodie abschaffen, und künftig nur noch psalmodiren lassen.}
         {}
  \chunks{Sie werden, verehrte Frau, noch heute, sobald von „Zukunftsmusik” die Rede ist, nichts Anderes vernehmen als diese Sätze. Bedenken Sie, mit welch machtvoller Nachhaltigkeit diese absurde Verleumdung aufrechterhalten und verbreitet worden sein muß, da neben der wirklichen und populären Verbreitung meiner Opern sie fast in der ganzen europäischen Presse, sobald mein Name erwähnt wird, sofort als ebenso unangefochten wie unwiderlegbar, mit stets neu verjüngter Kraft, auftritt.}
         {}
  \chunks{Da mir so unsinnige Theorien zugeschrieben werden konnten, mußten natürlich auch die Musikwerke, welche aus ihnen hervorgegangen, von widerlichster Beschaffenheit sein: ihr Erfolg mochte sein, welcher er wollte, immer blieb die Presse dabei, meine Musik müsse so abscheulich sein wie meine Theorie. Hierauf war nun der Nachdruck zu legen. Die eigentliche gebildete Intelligenz mußte für diese Ansicht gewonnen werden. Dies ward durch einen Wiener Juristen erreicht, welcher großer Musikfreund und Kenner der Hegelschen Dialektik war, außerdem aber durch seine, wenn auch zierlich verdeckte jüdische Abkunft besonders zugänglich befunden wurde. Auch Er war einer von Denjenigen, welche sich anfänglich mit fast enthusiastischer Neigung für mich erklärt hatten: seine Umtaufe geschah so plötzlich und gewaltsam, daß ich darüber völlig erschrocken war. Dieser schrieb nun ein Libell über das „Musikalisch-Schöne”, in welchem er für den allgemeinen Zweck des Musikjudenthums mit außerordentlichem Geschick verfuhr. Zunächst täuschte er durch eine höchst zierliche dialektische Form, welche ganz nach feinstem philosophischen Geiste aussah, die gesammte Wiener Intelligenz bis zu der Annahme, es sei denn wirklich einmal ein Prophet aus ihr hervorgegangen: und dieses war die beabsichtigte Hauptwirkung. Denn was er mit dieser eleganten dialektischen Färbung überzog, waren die trivialsten Gemeinplätze, wie sie mit einem Anschein von Bedeutsamkeit nur auf einem Gebiete sich ausbreiten können, auf welchem, wie auf dem der Musik, von jeher eben nur erst noch gefaselt worden war, sobald darüber ästhetisirt wurde. Es war gewiß kein Kunststück, auch für die Musik das „Schöne” als Hauptpostulat hinzustellen: brachte der Autor dies in der Art zu Stande, daß Alles über diese geniale Weisheit erstaunte, so gelang nun aber auch das allerdings Schwerere, nämlich die moderne jüdische Musik als die eigentliche „schöne” Musik aufzustellen; und zur stillschweigenden Anerkennung dieses Dogmas gelangte er ganz unvermerklich, indem er der Reihe Haydns, Mozarts und Beethovens so recht wie natürlich, Mendelssohn anschloß, ja – wenn man seine Theorie vom „Schönen” recht versteht, diesem Letzteren eigentlich die wohlthuende Bedeutung zusprach, das durch seinen unmittelbaren Vorgänger, Beethoven, einigermaßen in Confusion gerathene Schönheitsgewebe glücklich wieder arrangirt zu haben. War Mendelssohn so auf den Thron erhoben, was namentlich auch dadurch mit Manier zu bewerkstelligen war, daß man ihm einige christliche Notabilitäten, wie Robert Schumann zur Seite stellte, so war nun auch manches Weitere im Reiche der modernen Musik noch glaublich zu machen. Vor Allem aber war jetzt der schon angedeutete Hauptzweck der ganzen ästhetischen Unternehmung erreicht: der Verfasser hatte sich durch sein geistreiches Libell in allgemeinen Respect gesetzt, und sich hierdurch eine Stellung gemacht, welche ihm Bedeutung gab, wenn er, als angestaunter Aesthetiker, nun im gelesensten politischen Blatte auch als Recensent auftrat, und jetzt mich und meine künstlerischen Leistungen für rein null und nichtig erklärte. Daß ihn hierin der große Beifall, den meine Werke beim Publicum fanden, gar nicht beirrte, mußte ihm nur einen um so größeren Nimbus geben, und nebenbei erreichte er (oder auch: man erreichte durch ihn), daß, wenigstens so weit als Zeitungen in der Welt gelesen werden, eben dieser Ton über mich zum Styl geworden ist, welchen überall anzutreffen Sie, verehrteste Frau, so sehr verwunderte. Von Nichts als meiner Verachtung aller großen Tonmeister, meiner Feindschaft gegen die Melodie, von meinem gräulichen Componiren, kurz von „Zukunftsmusik” war nur noch die Rede: von jenem Artikel über „das Judenthum in der Musik” tauchte aber nie wieder das Mindeste auf. Dieser wirkte dagegen, wie an allen so seltsamen und plötzlichen Bekehrungswerken zu ersehen ist, desto erfolgreicher im Geheimen: er ward das Medusenhaupt, das sofort Jedem vorgehalten wurde, in welchem sich eine unbedachte Regung für mich zeigte.}
         {}
  \chunks{Wirklich nicht unbelehrend für die Culturgeschichte unsrer Tage dürfte es sein, diese sonderbaren Bekehrungswerke näher zu verfolgen, da sich hierdurch auf dem bisher von den Deutschen so ruhmvoll eingenommenen Gebiete der Musik eine seltsam verzweigte, und aus den unterschiedlichsten Elementen zusammengefügte Partei begründet hat, welche sich Impotenz und Unproductivität gegenseitig geradesweges versichert zu haben scheint.}
         {}
  \chunks{Sie werden, verehrte Frau, nun zunächst zwar fragen, wie es denn kam, daß die unläugbaren Erfolge, welche mir zu Theil wurden, und die Freunde, welche meine Arbeiten mir doch ganz offenbar gewannen, in keiner Weise zur Bekämpfung jener feindseligen Machinationen verwendet werden konnten?}
         {}
  \chunks{Dies ist nicht ganz leicht und kürzlich zu beantworten. Vernehmen Sie aber zunächst, wie es meinem größten Freunde und eifrigsten Für-Streiter, Franz Liszt erging. Gerade durch das großherzige Selbstvertrauen, welches er in Allem zeigte, lieferte er dem vorsichtig lauernden, und aus der geringfügigsten Nebensächlichkeit Gewinn ziehenden Gegner solche Waffen, wie gerade dieser sie brauchte. Was der Gegner so angelegentlich wünschte, die Secretirung der ihm so ärgerlichen Judenthumsfrage, war auch Liszt angenehm, natürlich aber aus dem entgegengesetzten Grunde, einem ehrlichen Kunststreite eine erbitternde persönliche Beziehung fernzuhalten, während Jenem daran lag, das Motiv eines unehrlichen Kampfes, den Erklärungsgrund der uns betreffenden Verleumdungen, verdeckt zu halten. Somit blieb dieses Ferment der Bewegung auch unsrerseits unberührt. Dagegen war es ein jovialer Einfall Liszts, den uns beigelegten Spottnamen der „Zukunftsmusiker”, in der Bedeutung, wie dies einst von den „gueux” der Niederlande geschah, zu acceptiren. Geniale Züge, wie dieser meines Freundes, waren dem Gegner höchst willkommen: er brauchte nun in diesem Punkte kaum mehr noch zu verleumden, und mit dem „Zukunftsmusiker” war jetzt dem feurig lebenden und schaffenden Künstler recht bequem beizukommen. Mit dem Abfalle eines bisher warm ergebenen Freundes, eines großen Violinvirtuosen, auf welchen das Medusenschild doch endlich auch gewirkt haben mochte, trat jene wüthende Agitation gegen den nach allen Seiten hin großmüthig unbesorgten Franz Liszt ein, welche ihm endlich die Enttäuschung und Verbitterung bereitete, in denen er seinen schönsten Bemühungen, der Musik in Weimar eine fördernde Stätte zu bereiten, für immer ein Ziel steckte.}
         {}
  \chunks{Sind Sie, verehrte Frau, nun über die Verfolgungen, denen seinerseits unser großer Freund ausgesetzt war, weniger verwundert, als über diejenigen, welche mich betroffen haben? – Vielleicht würde es Sie dann täuschen, daß Liszt allerdings durch den Glanz seiner äußerlichen Künstlerlaufbahn den Neid, namentlich der steckengebliebenen deutschen Collegen, auf sich gezogen hatte, außerdem aber durch sein Aufgeben der Virtuosenlaufbahn, und durch sein bis dahin nur vorbereitetes Auftreten als schaffender Tonsetzer, einen leicht auftauchenden, und daher vom Neide wiederum leicht zu nährenden Zweifel an seiner Berufung hierzu, in ziemlich begreiflicher Weise geweckt hat. Ich glaube jedoch mit Dem, was ich später noch berühren werde, nachweisen zu können, daß im tiefsten Grunde hier diese Zweifel nicht minder, als dort meine angeblichen Theorien, eben nur den Vorwand zu dem Verfolgungskriege abgaben: wie auf diese, so genügte es auf jene genauer hinzublicken und sie, mit dem richtigen Eindrucke von unsrem Schaffen, in Erwägung zu ziehen, so stand bald die Frage auf einem ganz anderen Punkte; da konnte dann geurtheilt, discutirt, für und wider gesprochen werden: am Ende wäre Etwas dabei herausgekommen. Aber gerade davon war nicht die Rede, ja, eben dieses nähere Beachten der neuen Erscheinungen wollte man nicht aufkommen lassen; sondern mit einer Gemeinheit des Ausdruckes und der Insinuation, wie es sich in keinem ähnlichen Falle nur je gezeigt hat, ward in der großen weiten Presse geschrieen und getobt, daß an ein menschenwürdiges Zuwortekommen gar nicht zu denken war. Und deßhalb versichere ich Sie: auch was Liszt widerfuhr, rührt von der Wirkung jenes Artikels über „das Judenthum in der Musik” her.}
         {}
  \chunks{Auch uns ging dies jedoch nicht sobald auf. Es giebt zu jeder Zeit so viele Interessen, welche zum Widerspruche gegen neue Erscheinungen, ja zur äußersten Verketzerung alles darin Enthaltenen bestimmen, daß auch wir hier eben nur mit der Trägheit und gestörten Kunstgeschäftsbequemlichkeit zu thun zu haben glauben konnten. Da die Anfeindungen sich vor Allem in der Presse, und zwar in der einflußreichen großen politischen Zeitungspresse, kundgaben, vermeinten namentlich diejenigen unsrer Freunde, welche die hierdurch gestörte Unbefangenheit des Publicums dem nun erfolgenden Auftreten Liszts als Instrumentalcomponist gegenüber besorgt machte, zur Gegenwirksamkeit schreiten zu müssen: einige Ungeschicklichkeiten abgerechnet, welche hierbei begangen wurden, zeigte es sich aber bald, daß selbst die besonnenste Besprechung einer Lisztschen Composition keinen Zugang zu den größeren Zeitungen fand, sondern daß hier Alles besetzt und im feindseligen Sinne in Beschlag genommen war. Wer wird nun im Ernste glauben wollen, daß sich in dieser Haltung der großen Zeitungen eine Besorgniß des Schadens aussprach, welchen etwa eine neue Kunstrichtung dem guten deutschen Kunstgeschmacke bringen könnte? Ich erlebte es mit der Zeit, daß in einem solchen geachteten Blatte es mir unmöglich werden sollte, Offenbachs in der ihm gebührenden Weise zu erwähnen: wer vermag hier an Sorge für den deutschen Kunstgeschmack zu denken? So weit war es eben gekommen: wir waren von der deutschen großen Presse vollständig ausgeschlossen. Wem gehört aber diese Presse? Unsre Liberalen und Fortschrittsmänner haben es empfindlich zu büßen, von den altconservativen Gegenparteien mit dem Judenthum und seinen specifischen Interessen in Einen Topf geworfen zu werden: wenn die römischen Ultras fragen, wie denn eine nur von den Juden dirigirte Presse berechtigt sein sollte, über christliche Kirchenangelegenheiten mitzusprechen, so liegt hierin ein fataler Sinn, der jedenfalls sich auf die richtige Kenntniß der Abhängigkeitsverhältnisse jener großen Zeitungen stützt.}
         {}
  \chunks{Das Sonderbare hierbei ist, daß diese Kenntniß auch Jedermann offenliegt; denn wer hat nicht seine Erfahrung davon gemacht? Ich kann nicht beurtheilen, wie weit dieses factische Verhältniß sich auch auf die größeren politischen Angelegenheiten erstreckt, wiewohl die Börse den Fingerzeig hierzu mit ziemlicher Offenheit giebt: auf diesem, dem ehrlosesten Geschwätze preisgegebenen Gebiete der Musik herrscht bei Einsichtsvollen gar kein Zweifel, daß hier Alles einer höchst merkwürdigen Ordensregel unterworfen ist, deren Befolgung in den weitestverzweigten Kreisen, und mit der übereinstimmendsten Genauigkeit, auf eine höchst energische Organisation und Leitung schließen läßt. In Paris fand ich zu meinem Erstaunen, daß namentlich auch diese sorgsamste Leitung gar kein Geheimniß war: Jeder weiß dort die wunderlichsten Züge davon zu berichten, namentlich im Betreff der bis in das Kleinlichste gehenden Sorge, das Geheimniß, da es nun doch einmal durch zu viele betheiligte Mitwisser der Unverschwiegenheit ausgesetzt war, wiederum dadurch wenigstens vor öffentlicher Denunciation zu bewahren, daß auch jedes noch so winzige Löchelchen, durch welches es in ein Journal dringen könnte, verstopft würde, und sei dies selbst durch eine Visitenkarte im Schlüsselloche eines Dachkämmerchens. Hier gehorchte denn auch Alles wie in der bestdisciplinirten Armee während der Schlacht: Sie lernten dieses gegen mich gerichtete Pelotonfeuer der Pariser Presse kennen, welches die Sorge für den guten Kunstgeschmack ihr commandirte. – In London traf ich seinerzeit in diesem Punkte größere Offenheit an. Ueberfiel mich der Musikkritiker der Times (ich bitte zu bedenken, von welchem kolossalen Weltblatte ich Ihnen hier erzähle!) bei meiner Ankunft sofort mit einem Hagel von Insulten, so genirte Herr Davison sich im Verlaufe seiner Ergießungen nicht weiter, mich als Lästerer der größten Componisten ihres Judenthums wegen, dem öffentlichen Abscheu anzuempfehlen. Mit dieser Aufdeckung hatte er allerdings bei dem englischen Publicum für sein Ansehen mehr zu gewinnen, als zu verlieren, einerseits der großen Verehrung wegen, welche Mendelssohn gerade dort genießt, andrerseits vielleicht aber auch wegen des eigenthümlichen Charakters der englischen Religion, welche Kennern mehr auf dem Alten, als auf dem Neuen Testamente zu fußen scheint. – Nur in Petersburg und Moskau fand ich das Terrain der musikalischen Presse von der Judenschaft noch vernachlässigt: dort erlebte ich das Wunder, zum ersten Male auch von den Zeitungen ganz so aufgenommen zu werden wie vom Publicum, dessen gute Aufnahme mir überhaupt die Juden nirgends noch hatten verderben können, außer in meiner Vaterstadt Leipzig, wo das Publicum mir einfach gänzlich wegblieb.}
         {}
  \chunks{Durch die lächerlichen Seiten der Sache bin ich bei dieser Mittheilung jetzt fast in einen scherzhaften Ton verfallen, den ich nun aber aufgeben muß, wenn ich es mir gestatten will, Sie, verehrte Frau, schließlich noch auf die sehr ernste Seite derselben aufmerksam zu machen: und diese beginnt auch vielleicht für Sie genau da, wo wir von meiner verfolgten Person absehen, um die Wirkung jener merkwürdigen Verfolgung, so weit sie sich auf unsren Kunstgeist selbst erstreckt, in das Auge zu fassen.}
         {}
  \chunks{Um diese Richtung einzuschlagen, habe ich zunächst mein persönliches Interesse noch einmal im Besonderen zu berühren. Ich sagte gelegentlich zuletzt, die von Seiten der Juden mir widerfahrene Verfolgung habe bisher mir noch nicht das Publicum, welches überall mit Wärme mich aufnahm, entfremden können. Dieses ist richtig. Jedoch muß ich dem nun hinzufügen, daß jene Verfolgung allerdings geeignet ist, mir die Wege zum Publicum, wenn nicht zu verschließen, so doch derart zu erschweren, daß endlich wohl auch nach dieser Seite hin der Erfolg der feindlichen Bemühungen vollständig zu werden versprechen dürfte. Bereits erleben Sie, daß, nachdem meine früheren Opern fast überall auf den deutschen Theatern sich Bahn gebrochen haben und dort mit stetem Erfolge gegeben worden sind, jedes meiner neueren Werke auf ein träges, ja feindselig ablehnendes Verhalten dieser selben Theater stößt: meine früheren Arbeiten waren nämlich schon vor der Judenagitation auf die Bühne gedrungen, und ihrem Erfolge war nicht mehr Viel anzuhaben. Nun aber hieß es, meine neuen Arbeiten seien nach den von mir seitdem veröffentlichten ‚unsinnigen‘ Theorien verfaßt, ich sei damit aus meiner früheren Unschuld gefallen, und kein Mensch könne meine Musik jetzt mehr anhören. Wie nun das ganze Judenthum nur durch die Benutzung der Schwächen und der Fehlerhaftigkeit unsrer Zustände Wurzel unter uns fassen konnte, so fand die Agitation auch hier sehr leicht den Boden, auf welchem – unrühmlich genug für uns! – Alles zu ihrem endlichen Erfolge vorgebildet liegt. In welchen Händen ist die Leitung unsrer Theater, und welche Tendenz befolgen diese Theater? Hierüber habe ich mich öfters und zur Genüge ausgesprochen, zuletzt auch noch in meiner größeren Abhandlung über „Deutsche Kunst und deutsche Politik” die weitverzweigten Gründe des Verfalles unsrer theatralischen Kunst ausführlicher bezeichnet. Glauben Sie, daß ich damit in den betreffenden Sphären mich beliebt gemacht hätte? Nur mit größter Abneigung, sie haben dies bewiesen, gehen jetzt die Administrationen der Theater an die Aufführung eines neuen Werkes von mir:[5] sie könnten aber hierzu gezwungen werden durch die meinen Opern allgemein günstige Haltung des Publicums; wie willkommen muß ihnen nun der Vorwand sein, welcher so leicht sich daraus ziehen läßt, daß meine neueren Arbeiten doch so allgemein in der Presse, und noch dazu im einflußreichsten Theile derselben bestritten wären? Hören Sie nicht schon jetzt aus Paris die Frage aufwerfen, warum man denn das an und für sich so schwierige Wagniß einer Uebersiedelung meiner Opern nach Frankreich glaube betreiben zu müssen, da meine künstlerische Bedeutung ja nicht einmal in der Heimat anerkannt sei? – Dieses Verhältniß erschwert sich nun aber um so mehr, als ich wirklich meine neueren Arbeiten keinem Theater anbiete, sondern im Gegentheil mir vorbehalten muß, bisher noch nie für nöthig gehaltene Bedingungen an meine etwa gewünschte Einwilligung zur Aufführung eines neuen Werkes zu knüpfen, nämlich die Erfüllung von Forderungen, welche mich einer wirklich correkten Darstellung desselben versichern sollen.[6] Und hiermit berühre ich denn nun die ernstlichste Seite des nachtheiligen Erfolges der Einmischung des jüdischen Wesens in unsre Kunstzustände.}
         {}
  \chunks{In meinem voranstehenden älteren Aufsatze zeigte ich schließlich, daß es die Schwäche und Unfähigkeit der nachbeethovenschen Periode unsrer deutschen Musikproduction war, welche die Einmischung der Juden in dieselbe zuließ: ich bezeichnete alle diejenigen unsrer Musiker, welche in der Verwischung des großen plastischen Styles Beethovens die Ingredienzien für die Zubereitung der neueren gestaltungslosen, seichten, mit dem Anscheine der Solidität matt sich übertünchenden Manier fanden, und in dieser nun ohne Leben und Streben mit duseligem Behagen so weiter hin componirten, als in dem von mir geschilderten Musikjudenthum durchaus mitinbegriffen, möchten sie einer Nationalität angehören, welcher sie wollten. Diese eigenthümliche Gemeinde ist es, welche gegenwärtig so ziemlich Alles in sich faßt, was Musik componirt und – leider auch! – dirigirt. Ich glaube, daß Manche von ihnen durch meine Kunstschriften ehrlich confus gemacht und erschreckt worden sind: ihre redliche Verwirrung und Betroffenheit war es, welcher die Juden, im Zorn über meinen obigen Artikel, sich bemächtigen, um jede anständige Discussion meiner anderweitigen theoretischen Thesen sofort abzuschneiden, da zu der Möglichkeit einer solchen von Seiten ehrlicher deutscher Musiker anfänglich sich beachtenswerthe Ansätze zeigten. Mit den paar genannten Schlagwörtern ward jede befruchtende, erklärende, läuternde und bildende Erörterung und gegenseitige Verständigung hierüber niedergehalten. – Derselbe schwächliche Geist lebte nun aber in Folge der Verwüstungen, welche die Hegelsche Philosophie in den zu abstrakter Meditation so geneigten deutschen Köpfen angerichtet hatte, auch auf diesem, wie auf dem zu ihm gehörigen Gebiete der Aesthetik, nachdem Kants große Idee, von Schiller so geistvoll zur Begründung ästhetischer Ansichten über das Schöne benutzt, einem wüsten Durcheinander von dialektischen Nichtssäglichkeiten Platz hatte machen müssen. Selbst von dieser Seite traf ich jedoch anfänglich auf eine Neigung, mit redlichem Willen auf die in meinen Kunstschriften niedergelegten Ansichten einzugehen. Jenes erwähnte Libell des Dr. Hanslick in Wien über das „Musikalisch-Schöne”, wie es mit bestimmter Absicht verfaßt worden, ward aber auch mit größter Hast schnell zu solcher Berühmtheit gebracht, daß es einem gutartigen, durchaus blonden deutschen Aesthetiker, Herrn Vischer, welcher sich bei der Ausführung eines großen Systems mit dem Artikel „Musik” herumzuplagen hatte, nicht wohl zu verdenken war, wenn er sich der Bequemlichkeit und Sicherheit wegen mit dem so sehr gepriesenen Wiener Musikästhetiker associirte: er überließ ihm die Ausführung dieses Artikels, von dem er Nichts zu verstehen bekannte, für sein großes Werk.[7] So saß denn die musikalische Judenschönheit mitten im Herzen eines vollblutig germanischen Systems der Aesthetik, was auch zur Vermehrung der Berühmtheit seines Schöpfers um so mehr beitrug, als es jetzt überlaut in den Zeitungen gepriesen, seiner großen Unkurzweiligkeit wegen aber von Niemand gelesen ward. Unter der verstärkten Protection durch diese neue, noch dazu ganz christlich-deutsche Berühmtheit, ward nun auch die musikalische Judenschönheit zum völligen Dogma erhoben; die eigenthümlichsten und schwierigsten Fragen der Aesthetik der Musik, über welche die größten Philosophen, sobald sie etwas wirklich Gescheidtes sagen wollten, sich stets nur noch mit muthmaßender Unsicherheit geäußert hatten, wurden von Juden und übertölpelten Christen jetzt mit einer Sicherheit zur Hand genommen, daß demjenigen, der sich hierbei wirklich Etwas denken, und namentlich den überwältigenden Eindruck der Beethovenschen Musik auf sein Gemüth sich erklären wollte, etwa so zu Muthe werden mußte, als hörte er der Verschacherung der Gewänder des Heilands am Fuße des Kreuzes zu, – worüber der berühmte Bibelforscher David Strauß vermuthlich eben so geistvoll erläuternd, wie über die neunte Symphonie Beethovens, sich auslassen dürfte.}
         {}
  \chunks{Dieses Alles mußte nun endlich den weitergehenden Erfolg haben, daß, wenn im Gegensatze zu diesem eben so rührigen, als unproductiven Getreibe, der Versuch zu einer Erkräftigung des immer mehr erschlaffenden Kunstgeistes gemacht werden sollte, wir nicht nur auf die natürlichen, zu jeder Zeit hiergegen sich einstellenden Hindernisse, sondern auch auf eine vollständig organisirte Opposition trafen, als welche die in ihr begriffenen Elemente sich sogar einzig nur thätig zu zeigen vermochten. Schienen wir verstummt und resignirt, so ging nämlich im andren Lager eigentlich gar Nichts vor, was wie ein Wollen, Streben und Hervorbringen anzusehen war: vielmehr ließ man gerade auch von Seiten der Bekenner der reinen Judenmusikschönheit Alles geschehen, und jede neue Calamität à la Offenbach über das deutsche Kunstwesen hereinbrechen, ohne sich auch nur zu rühren, was Sie allerdings nun „selbstverständlich” finden werden. Wurde dagegen Jemand, wie eben ich, durch irgend eine ermuthigende Gunst der Umstände veranlaßt, dargebotene künstlerische Kräfte zur Hand zu nehmen, um sie zu energischer Bethätigung anzuleiten, so vernahmen Sie ja wohl, verehrte Frau, welches Geschrei dies allseitig hervorrief? Da kam Kraft und Feuer in die Gemeinde des modernen Israel! Vor Allem fiel hierbei stets auch die Geringschätzung, der ganze unehrerbietige Ton auf, welchen, wie ich glaube, nicht nur die blinde Leidenschaftlichkeit, sondern die sehr hellsehende Berechnung der unvermeidlichen Wirkung davon auf die Beschützer meiner Unternehmungen eingab; denn wer fühlt sich nicht endlich von dem wegwerfenden Tone, mit welchem allgemein über Denjenigen, dem man vor aller Welt wahre Verehrung und hohes Vertrauen erweist, gesprochen wird, betroffen? Ueberall und in jedem Verhältnisse, welches zu complicirten Unternehmungen verwendet werden soll, sind die ganz natürlichen Elemente der Mißgunst der Unbetheiligten (oder auch der zu nahe Betheiligten) vorhanden: wie leicht wird es nun durch jenes geringschätzige Benehmen der Presse diesen Allen gemacht, das Unternehmen selbst im Auge seiner Gönner bedenklich erscheinen zu lassen? Kann so Etwas einem vom Publicum gefeierten Franzosen in Frankreich, einem acclamirten italienischen Tonsetzer in Italien begegnen? Was nur einem Deutschen in Deutschland widerfahren konnte, war so neu, daß die Gründe davon jedenfalls erst zu untersuchen sind. Sie, verehrte Frau, verwunderten sich darüber; die bei diesem anscheinenden Kunstinteressenstreite übrigens Unbetheiligten, welche sonst jedoch Gründe haben, Unternehmungen, wie sie von mir ausgehen, zu verhindern, verwundern sich aber nicht, sondern finden Alles recht natürlich.[8]}
         {}
  \chunks{Der Erfolg hiervon ist also: immer entschiedener durchgesetzte Verhinderung jeder Unternehmung, welche meinen Arbeiten und meinem Wirken einen Einfluß auf unsre theatralischen und musikalischen Kunstzustände verschaffen könnte.}
         {}
  \chunks{Ist hiermit Etwas gesagt? – Ich glaube: Viel; und vermeine hierbei ohne Anmaßung mich vernehmen zu lassen. Daß ich meinem Wirken eine wesentliche Bedeutung beilegen darf, ersehe ich daraus, wie ernstlich es vermieden wird, auf diejenigen meiner Veröffentlichungen einzugehen, zu welchen ich in diesem Betreff gelegentlich veranlaßt worden bin.}
         {}
  \chunks{Ich erwähnte, wie anfänglich, ehe die so sonderbar ihren Grund verheimlichende Agitation der Juden gegen mich eintrat, die Ansätze zu einer ehrlich deutsch geführten Behandlung und Erwägung der von mir in meinen Kunstschriften niedergelegten Ansichten sich zeigten. Nehmen wir an, jene Agitation wäre nun nicht eingetreten, oder sie hätte, wie billig, sich ebenfalls offen und ehrlich auf ihre nächste Veranlassung beschränkt, so hätten wir uns wohl zu fragen, wie dann, nach der Analogie gleichartiger Vorgänge im ungestörten deutschen Culturleben, die Sache sich gestaltet haben würde. Ich bin nicht der optimistischen Meinung, daß hierbei sehr Viel herausgekommen wäre; wohl aber wäre Etwas zu erwarten gewesen, und jedenfalls etwas Anderes, als das eingetretene Ergebniß. Verstehen wir es recht, so war, wie für die poetische Litteratur, auch für die Musik die Periode der Sammlung eingetreten, um die Hinterlassenschaft der unvergleichlichen Meister, welche in dicht an einander sich schließender Reihe die große deutsche Kunstwiedergeburt selbst darstellen, zu einem Gemeingut der Nation, der Welt verwerthen zu sollen. In welchem Sinne diese Verwerthung sich bestimmen würde, das war die Frage. Am entscheidendsten gestaltete sie sich für die Musik: denn hier war namentlich durch die letzten Perioden des Beethovenschen Schaffens eine ganz neue Phase der Entwickelung dieser Kunst eingetreten, welche alle von ihr bisher gehegten Ansichten und Annahmen durchaus überbot. Die Musik war unter der Führung der italienischen Gesangsmusik zur Kunst der reinen Annehmlichkeit geworden: die Fähigkeit, sich die gleiche Bedeutung der Kunst Dantes und Michel Angelos zu geben, leugnete man damit durchaus ab, und verwies sie somit in einen offenbar niedereren Rang der Künste überhaupt. Es war daher aus dem großen Beethoven eine ganz neue Erkenntniß des Wesens der Musik zu gewinnen, die Wurzel, aus welcher sie gerade zu dieser Höhe und Bedeutung erwachsen, sinnvoll durch Bach auf Palestrina zu verfolgen, und somit ein ganz anderes System für ihre ästhetische Beurtheilung zu begründen, als dasjenige sein konnte, welches sich auf die Kenntnißnahme einer von diesen Meistern weit abliegenden Entwickelung der Musik stützte.}
         {}
  \chunks{Das richtige Gefühl hiervon war ganz instinctiv in den deutschen Musikern dieser Periode lebendig, und ich nenne Ihnen hier Robert Schumann als den sinnvollsten und begabtesten dieser Musiker. An dem Verlaufe seiner Entwickelung als Componist läßt sich recht ersichtlich der Einfluß nachweisen, welchen die von mir bezeichnete Einmischung des jüdischen Wesens auf unsere Kunst ausübte. Vergleichen Sie den Robert Schumann der ersten, und den der zweiten Hälfte seines Schaffens: dort plastischer Gestaltungstrieb, hier Verfließen in schwülstige Fläche bis zur geheimnißvoll sich ausnehmenden Seichtigkeit. Dem entspricht es, daß Schumann in dieser zweiten Periode mißgünstig, mürrisch und verdrossen auf Diejenigen blickte, welchen er in seiner ersten Periode als Herausgeber der „Neuen Zeitschrift für Musik” so warm und deutsch liebenswürdig die Hand gereicht hatte. An der Haltung dieser Zeitschrift, in welcher Schumann (mit ebenfalls sehr richtigem Instincte) auch schriftstellerisch für die große uns obliegende Aufgabe sich bethätigte, können Sie gleichfalls ersehen, mit welchem Geiste ich mich zu berathen gehabt hätte, wenn ich mit ihm allein über die mich anregenden Probleme mich verständigen sollte: hier treffen wir wahrlich auf eine andere Sprache, als den endlich in unsre neue Aesthetik hinübergeleiteten dialektischen Judenjargon, und – ich bleibe dabei! – in dieser Sprache wäre es zu einem fördernden Einvernehmen gekommen. Was aber gab dem jüdischen Einflusse diese Macht? Leider ist eine Haupttugend des Deutschen auch der Quell seiner Schwächen. Das ruhige, gelassene Selbstvertrauen, das ihm bis zum Fernhalten alles peinigenden Seelenskrupels eigen bleibt, und so manche innig treue That aus seiner ungestört sich gleichen Natur hervortreibt, kann bei einem nur geringen Mangel an nöthigem Feuer leicht zu jener wunderlichen Trägheit umschlagen, in welche wir jetzt, unter der andauernden Verwahrlosung aller höheren Anliegen des deutschen Geistes in den machtvollen politischen Sphären, die meisten, ja fast alle dem deutschen Wesen ganz treu verbliebenen Geister versunken sehen. In diese Trägheit versank auch Robert Schumanns Genius, als es ihn belästigte, dem geschäftig unruhigen jüdischen Geiste Stand zu halten; es war ihm ermüdend, an tausend einzelnen Zügen, welche zunächst an ihn herantraten, sich stets deutlich machen zu sollen, was hier vorging. So verlor er unbewußt seine edle Freiheit, und nun erleben es seine alten, von ihm endlich gar verleugneten Freunde, daß er als einer der Ihrigen von den Musikjuden uns im Triumphe dahergeführt wird! – Nun, verehrte Freundin, dies wäre, so denke ich, ein Erfolg, der Etwas zu sagen hat? Seine Vorführung erspart uns jedenfalls die Beleuchtung geringfügigerer Unterjochungsfälle, welche in Folge dieses wichtigsten immer leichter hervorzurufen waren.}
         {}
  \chunks{Diese persönlichen Erfolge vervollständigen sich aber auf dem Gebiete des Associations- und Gesellschaftswesens. Auch hier zeigte sich der deutsche Geist noch seiner Anlage gemäß zur Bethätigung angeregt. Die Idee, welche ich Ihnen als die Aufgabe unsrer nachbeethovenschen Periode bezeichnete, vereinigte auch wirklich zum ersten Male eine immer größere Anzahl deutscher Musiker und Musikfreunde zu Zwecken, welche ihre natürliche Bedeutung durch das Erfassen jener Aufgabe erhielten. Es ist dem trefflichen Franz Brendel, der auch hierzu mit treuer Ausdauer die Anregung gab, und welchem dafür geringschätzig zu begegnen zum Tone der Judenblätter wurde, zum wahren Ruhme anzurechnen, nach dieser Seite hin das Nöthige ebenfalls erkannt zu haben. Das Gebrechen alles deutschen Associationswesens mußte aber auch hier um so eher sich herausstellen, als mit einem Vereine deutscher Musiker nicht etwa nur den machtvollen Sphären der staatlichen, von den Regierungen geleiteten Organisationen, wie mit anderen, zu gleicher Wirkungslosigkeit verurtheilten freien Vereinigungen es der Fall ist, sondern dabei noch den Interessen der allermächtigsten Organisation unsrer Zeit, der des Judenthumes, entgegengetreten wurde. Offenbar konnte ein großer Verein von Musikern nur auf dem praktischen Wege vorzüglichster Musteraufführungen für die Ausbildung des deutschen Musikstyles wichtiger Werke eine erfolgreiche Bethätigung ausüben; hierzu gehörten Mittel; der deutsche Musiker ist aber arm: wer wird ihm helfen? Gewiß nicht das Reden und Disputiren über Kunstinteressen, welches unter Vielen nie einen Sinn haben kann, und leicht zum Lächerlichen führt. Jene uns fehlende Macht gehörte aber dem Judenthum. Die Theater den Junkern und dem Coulissenjux, die Concertinstitute den Musikjuden: was blieb uns da noch übrig? Etwa ein kleines Musikblatt, das über den Ausfall der allzweijährlichen Zusammenkünfte Bericht gab.\ornamentbreak}
         {}
% * * *
  \chunks{Wie Sie sehen, verehrte Frau, bezeuge ich Ihnen hiermit den vollständigen Sieg des Judenthums auf allen Seiten; und wenn ich mich jetzt noch einmal laut darüber ausspreche, so geschieht dies wahrlich nicht in der Meinung, ich könnte der Vollständigkeit dieses Sieges noch in Etwas Abbruch thun. Da nun andrerseits meine Darstellung des Verlaufes dieser eigenthümlichen Culturangelegenheit des deutschen Geistes zu besagen scheint, dieses sei das Ergebniß der durch meinen früheren Artikel unter den Juden hervorgerufenen Agitation, so läge Ihnen vielleicht auch die neue verwunderungsvolle Frage darnach nicht fern, warum ich denn durch jene Herausforderung eben diese Agitation als Reaction hervorgerufen hätte?}
         {}
  \chunks{Ich könnte mich hierfür damit entschuldigen, daß ich zu diesem Angriffe nicht durch Erwägung der „causa finalis”, sondern einzig durch den Antrieb der „causa efficiens” (wie der Philosoph sich ausdrückt) bestimmt worden sei. Gewiß hatte ich schon bei der Abfassung und Veröffentlichung jenes Aufsatzes Nichts weniger im Sinne, als den Einfluß der Juden auf unsre Musik mit Aussicht auf Erfolg noch zu bekämpfen: die Gründe ihrer bisherigen Erfolge waren mir damals bereits so klar, daß es mir jetzt, nach über achtzehn Jahren, gewissermaßen zur Genugthuung dient, durch die Wiederveröffentlichung desselben dieses bezeugen zu können. Was ich damit bezwecken wollte, könnte ich daher nicht klar bezeichnen, dagegen nur eben mich darauf berufen, daß die Einsicht in den unvermeidlichen Verfall unsrer Musikzustände mir die innere Nöthigung zur Bezeichnung der Ursachen davon auferlegte. Vielleicht lag es aber doch auch meinem Gefühle nahe, eine hoffnungsreiche Annahme noch damit zu verbinden: dies enthüllt Ihnen die Schlußapostrophe des Aufsatzes, mit welcher ich mich an die Juden selbst wende.}
         {}
  \chunks{Wie nämlich von humanen Freunden der Kirche eine heilsame Reform derselben durch Berufung an den unterdrückten niederen Klerus als möglich gedacht worden ist, so faßte auch ich die großen Begabungen des Herzens wie des Geistes in das Auge, die aus dem Kreise der jüdischen Societät mir selbst zu wahrer Erquickung entgegengekommen sind. Gewiß bin ich auch der Meinung, daß Alles, was das eigentliche deutsche Wesen von dorther bedrückt, in noch viel schrecklicherem Maaße auf dem geist- und herzvollen Juden selbst lastet. Mich dünkt es, als ob ich damals Anzeichen davon wahrnahm, daß meine Anrufung Verständniß und tiefe Erregung hervorgerufen hatte. Ist Abhängigkeit in jeder Lage ein großes Uebel und Hinderniß der freien Entwickelung so scheint die Abhängigkeit der Juden unter sich aber ein knechtisches Elend von alleräußerster Härte zu sein. Es mag dem geistreichen Juden, da man nun einmal nicht nur mit uns, sondern in uns zu leben sich entschlossen hat, von der aufgeklärteren Stammgenossenschaft Vieles gestattet und nachgesehen werden: die besten, so sehr erheiternden Judenanecdoten werden von ihnen uns erzählt; auch nach anderen Seiten hin, über uns, wie über sich, kennen wir sehr unbefangene, und somit jedenfalls erlaubt dünkende Auslassungen von ihnen. Aber einen vom Stamme Geächteten in Schutz zu nehmen, das muß jedenfalls den Juden als geradesweges todeswürdiges Verbrechen gelten. Mir sind hierüber rührende Erfahrungen zu Theil geworden. Um Ihnen aber diese Tyrannei selbst zu bezeichnen, diene ein Fall für viele. Ein offenbar sehr begabter, wirklich talent- und geistvoller Schriftsteller jüdischer Abkunft, welcher in das eigenthümlichste deutsche Volksleben wie eingewachsen erscheint, und mit dem ich längere Zeit auch über den Punkt des Judenthumes mannigfach verkehrte, lernte späterhin meine Dichtungen: „Der Ring des Nibelungen” und „Tristan und Isolde” kennen; er sprach sich darüber mit solch anerkennender Wärme und solch deutlichem Verständniß aus, daß die Aufforderung meiner Freunde, zu welchen er gesprochen hatte, wohl nahe lag, seine Ansicht über diese Gedichte, welche von unsren litterarischen Kreisen so auffallend ignorirt würden, auch öffentlich darzulegen. Dies war ihm unmöglich! –}
         {}
  \chunks{Begreifen Sie, verehrte Frau, aus diesen Andeutungen, daß, wenn ich auch diesmal nur Ihrer Frage nach dem räthselhaften Grunde der mir widerfahrenden Verfolgungen, namentlich der Presse, antwortete, ich meiner Antwort dennoch vielleicht nicht diese, fast ermüdende, Ausdehnung gegeben haben würde, wenn nicht auch heute noch eine, allerdings fast kaum auszusprechende, im tiefsten Sinne mir liegende Hoffnung mich dabei angeregt hätte. Wollte ich dieser einen Ausdruck geben, so durfte ich sie vor Allem nicht auf eine fortgesetzte Verheimlichung meines Verhältnisses zu dem Judenthume begründet erscheinen lassen: diese Verheimlichung hat zu der Verwirrung beigetragen, in welcher sich heute fast jeder für mich theilnehmende Freund mit Ihnen befindet. Habe ich hierzu durch jenen früheren Pseudonym Anlaß, ja dem Feinde das strategische Mittel zu meiner Bekämpfung an die Hand gegeben, so mußte ich nun auch für meine Freunde Dasselbe enthüllen, was Jenen nur zu wohl bekannt war. Wenn ich annehme, daß nur diese Offenheit auch Freunde im feindlichen Lager, nicht sowohl mir zuführen, als zum eigenen Kampfe für ihre wahre Emancipation stärken könne, so ist es mir vielleicht zu verzeihen, wenn ein umfassender culturhistorischer Gedanke mir die Beschaffenheit einer Illusion verdeckt, welche unwillkürlich sich in mein Herz schmeichelt. Denn über Eines bin ich mir klar: so wie der Einfluß, welchen die Juden auf unser geistiges Leben gewonnen haben, und wie er sich in der Ablenkung und Fälschung unsrer höchsten Culturtendenzen kundgiebt, nicht ein bloßer, etwa nur physiologischer Zufall ist, so muß er auch als unläugbar und entscheidend anerkannt werden. Ob der Verfall unsrer Cultur durch eine gewaltsame Auswerfung des zersetzenden fremden Elementes aufgehalten werden könne, vermag ich nicht zu beurtheilen, weil hierzu Kräfte gehören müßten, deren Vorhandensein mir unbekannt ist. Soll dagegen dieses Element uns in der Weise assimilirt werden, daß es mit uns gemeinschaftlich der höheren Ausbildung unsrer edleren menschlichen Anlagen zureife, so ist es ersichtlich, daß nicht die Verdeckung der Schwierigkeiten dieser Assimilation, sondern nur die offenste Aufdeckung derselben hierzu förderlich sein kann. Sollte von dem, unsrer neuesten Aesthetik nach, so harmlos annehmlichen Gebiete der Musik aus von mir eine ernste Anregung hierzu gegeben worden sein, so würde dies vielleicht meiner Ansicht über die bedeutende Bestimmung der Musik nicht ungünstig erscheinen; und jedenfalls würden Sie, hochverehrte Frau, hierin eine Entschuldigung dafür erkennen dürfen, daß ich Sie so lange von diesem anscheinend so abstrusen Thema unterhielt.}
         {}
  \chunks{%
\begin{flushright}
Tribschen bei Luzern, Neujahr 1869.

Richard Wagner.
\end{flushright}}
         {}

%↑ À ce sujet, on peut trouver bien des choses à dire sur l’activité des acteurs juifs, par suite de certains faits récents ; je ne m’y arrêterai pas. Non seulement, les Juifs ont réussi à accaparer la scène, ils ont encore subtilisé au poète ses créations dramatiques. Un de leurs acteurs, célèbre dans les rôles de caractère, n’incarne plus du tout les personnages créés par Shakespeare, Schiller, etc., il leur substitue des créations tendancieuses de son cru, visant à l’effet, et donnant l’impression que produirait un tableau de la Crucification où le Christ, découpé, aurait été remplacé par un juif démagogue. La falsification de notre art confine au pire mensonge. C’est pour cette raison que Shakespeare et ses émules ne seront plus évoqués qu’au point de vue de leur utilisation conditionnelle au théâtre.
%↑ Pour quiconque a observé la tenue insolente et l’indifférence d’une assemblée juive à sa synagogue, au cours d’un service divin en musique, il est facile de comprendre qu’un compositeur d’opéra juif ne se sente pas blessé de retrouver la même chose chez un public de théâtre et soit capable de travailler sans dégoût pour lui, car elle doit lui paraître moins inconvenante ici que dans la maison de Dieu.
%↑ L’attitude adoptée à l’égard de ces deux compositeurs célèbres par les musiciens juifs, et en général par les juifs cultivés, est non moins caractéristique. Pour les partisans de Mendelssohn, ce grand compositeur d’opéras apparaît comme un épouvantail ; ils se rendent compte avec un orgueil délicat qu’il compromet le judaïsme vis-à-vis du compositeur plus raffiné, et leurs jugements sont en conséquence sans ménagement. En revanche, les partisans de ce compositeur sont plus circonspects sur le compte de Mendelssohn, en considérant avec plus d’envie que d’animosité la fortune qu’il a réalisée dans un monde musical « plus élevé ». À un troisième tiers, formé de ceux qui continuent de composer, il semble préférable d’éviter tout scandale parmi eux, afin de ne pas se compromettre et de voir leur production musicale faire son chemin sans éveiller l’attention ; les succès indiscutables du grand compositeur d’opéras sont en conséquence dignes d’attention pour eux, et cela serait vrai, s’il n’y avait beaucoup à admettre et à accepter comme « argent comptant ». Les juifs sont en vérité trop bien avisés pour méconnaitre au fond ce qu’il en est quant à eux.

%↑ Hierüber läßt sich nach den neueren Erfahrungen von der Wirksamkeit jüdischer Schauspieler allerdings noch Manches sagen, worauf ich hier im Vorbeigehen nur hindeute. Den Juden ist es seitdem nicht nur gelungen, auch die Schaubühne einzunehmen, sondern selbst dem Dichter seine dramatischen Geschöpfe zu escamotiren; ein berühmter jüdischer „Charakterspieler“ stellt nicht mehr die gedichteten Gestalten Shakespeare’s, Schiller’s u. s. w. dar, sondern substituirt diesen die Geschöpfe seiner eigenen effectvollen und nicht ganz tendenzlosen Auffassung, was dann etwa den Eindruck macht, als ob aus einem Gemälde der Kreuzigung der Heiland ausgeschnitten, und dafür ein demagogischer Jude hineingesteckt sei. Die Fälschung unsrer Kunst ist auf der Bühne bis zur vollendeten Täuschung gelungen, weßhalb denn auch jetzt über Shakespeare und Genossen nur noch in Betreff ihrer bedingungsweisen Verwendbarkeit für die Bühne gesprochen wird.
%↑ Ueber das neu-jüdische System, welches auf diese Eigenschaft der Mendelssohnschen Musik, wie zur Rechtfertigung dieser künstlerischen Verkommniß, entworfen worden ist, sprechen wir später.
%↑ Wer die freche Zerstreutheit und Gleichgiltigkeit einer jüdischen Gemeinde während ihres musikalisch ausgeführten Gottesdienstes in der Synagoge beobachtet hat, kann begreifen, warum ein jüdischer Operncomponist durch das Antreffen derselben Erscheinung bei einem Theaterpublicum sich gar nicht verletzt fühlt, und unverdrossen für dasselbe zu arbeiten vermag, da sie ihn hier sogar minder unanständig dünken muß als im Gotteshause.
%↑ Charakteristisch ist noch die Stellung, welche die übrigen jüdischen Musiker, ja überhaupt die gebildete Judenschaft, zu ihren beiden berühmtesten Componisten einnehmen. Den Anhängern Mendelssohns ist jener famose Operncomponist ein Gräuel: sie empfinden mit seinem Ehrgefühle, wie sehr er das Judenthum dem gebildeteren Musiker gegenüber compromittirt, und sind deshalb ohne alle Schonung in ihrem Urtheil. Bei weitem vorsichtiger äußert sich dagegen der Anhang dieses Componisten über Mendelssohn, mehr mit Neid, als mit offenbarem Widerwillen das Glück betrachtend, das er in der “gediegeneren” Musikwelt gemacht hat. Einer dritten Fraction, derjenigen der immer noch fortcomponirenden Juden, liegt es ersichtlich daran, jeden Scandal unter sich zu vermeiden, um sich überhaupt nicht bloßzustellen, damit ihr Musikproduciren ohne alles peinliche Aufsehen seinen bequemen Fortgang nehme: die immerhin unläugbaren Erfolge des großen Operncomponisten gelten ihnen denn doch für beachtenswerth, und Etwas müsse doch daran sein, wenn man auch Vieles nicht gutheißen und für “solid” ausgeben könnte. In Wahrheit, die Juden sind viel zu klug, um nicht zu wissen, wie es im Grunde mit ihnen steht!

%↑ Es wäre nicht unbelehrend und jedenfalls für unsre Kunstzustände bezeichnend, wenn ich mich Ihnen über das Verfahren näher ausließe, welches ich neuerdings, zu meinem wahren Erstaunen, von Seiten der beiden größten Theater, Berlins und Wiens, in Betreff meiner „Meistersinger” kennen lernen mußte. Es bedurfte in meinen Verhandlungen mit den Leitern dieser Hoftheater einiger Zeit, ehe ich aus den von ihnen hierbei angewendeten Kniffen ersah, daß es ihnen nicht allein darum zu thun war, mein Werk nicht geben zu dürfen, sondern auch zu verhindern, daß es auf anderen Theatern gegeben werde. Sie würden daraus deutlich ersehen müssen, daß es sich hierbei um eine wirkliche Tendenz handelt, und offenbar über das Erscheinen eines neuen Werkes von mir ein wahrer Schrecken empfunden wurde. Vielleicht unterhält es Sie, auch hierüber einmal etwas Näheres aus dem Bereiche meiner Erfahrungen zu vernehmen.
%↑ Nur dadurch, daß ich, für jetzt aus nothgedrungener Rücksicht auf meinen Verleger, diese Forderungen fallen ließ, konnte ich neuerdings das Dresdener Hoftheater zur Vornahme der Aufführung meiner „Meistersinger” bewegen.
%↑ Dieses theilte mir Herr Professor Vischer einst selbst in Zürich mit: in welchem Verhältniß die Mitarbeit des Herrn Hanslick als eine persönliche und unmittelbare herbeigezogen wurde, ist mir unbekannt geblieben. 
%↑ Sie können sich hiervon, und von der Art, wie die zuletzt von mir Bezeichneten den in meinem Betreff aufgebrachten Ton des Weiteren zu den Zwecken der Verhinderung jedes meine Unternehmungen fördernden Antheiles benutzen, einen recht genügenden Begriff verschaffen, wenn Sie das Feuilleton der heurigen Neujahrsnummer der „Süddeutschen Presse”, welche mir soeben aus München zugeschickt wird, zu durchlesen sich bemühen wollen. Herr Julius Fröbeldenuncirt mich da dem bayerischen Staatswesen ganz unbeirrt als den Gründer einer Secte, welche den Staat und die Religion abzuschaffen, dagegen alles Dieses durch ein Operntheater zu ersetzen und von ihm aus zu regieren beabsichtigt, außerdem aber auch Befriedigung „muckerhafter Gelüste” in Aussicht stellt. – Der verstorbene Hebbel bezeichnete mir einmal im Gespräche die eigenthümliche Gemeinheit des Wiener Komikers Nestroy damit, daß eine Rose, wenn dieser daran gerochen haben würde, jedenfalls stinken müßte. Wie sich die Idee der Liebe, als Gesellschaftsgründerin, im Kopfe eines Julius Fröbel ausnimmt, erfahren wir hier mit einem ähnlichen Effect. – Aber begreifen Sie, wie sinnvoll so Etwas wiederum auf die Erweckung des Ekels berechnet ist, mit welchem selbst der Verleumdete sich von der Bestrafung des Verleumders abwendet?
\end{paracol}
\end{document}
