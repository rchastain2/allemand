
% https://tex.stackexchange.com/a/505562/295527

%\documentclass[a6paper,pagesize,12pt]{book}
%\documentclass[a4paper,10pt]{book}
\documentclass[a4paper,10pt]{article}

\usepackage[margin=8mm]{geometry}
%\usepackage[svgnames]{xcolor}
\usepackage[french, main=german]{babel}
\usepackage{fontspec, microtype, paracol}

\defaultfontfeatures{Scale=MatchUppercase}
\babelfont{rm}
          [Scale=1.0,
           Language=Default,
           Ligatures={Common, TeX}
          ]{TeX Gyre Schola} % Or your font of choice.

\newfontfamily\symbolfont{DejaVu Sans}

%\pagenumbering{gobble}
%\setlength\columnseprule{0.4pt}
%\sloppy

\usepackage{verse}

\begin{document}
%\begin{center}
%{\LARGE \textbf{\textsc{Red svete maše}}} \\ \vspace{1cm}
%\textbf{Pristopne molitve (KLEČIMO)}
%\end{center}
\begin{paracol}{2}
\begin{leftcolumn}\begin{otherlanguage}{german}%
\settowidth{\versewidth}{Bestand noch Recht und Billigkeit.}%
\begin{verse}[\versewidth]%
\poemtitle{Zu meiner Zeit.}%
% https://www.lieder.net/lieder/get_text.html?TextId=123236
\noindent
Zu meiner Zeit, zu meiner Zeit,\\
Bestand noch Recht und Billigkeit.\\
Da wurden auch aus Kindern Leute,\\
Aus tugendhaften Mädchen Bräute;\\
Doch alles mit Bescheidenheit.\\
O gute Zeit, o gute Zeit!\\
Es ward kein Jüngling zum Verräter,\\
Und unsre Jungfern freiten später,\\
Sie reizten nicht der Mütter Neid.\\
O gute, Zeit, o gute Zeit!

Zu meiner Zeit, zu meiner Zeit\\
Befliß man sich der Heimlichkeit.\\
Genoß der Jüngling ein Vergnügen,\\
So war er dankbar und verschwiegen;\\
Doch jetzt entdeckt er's ungescheut.\\
O schlimme Zeit, o schlimme Zeit!\\
Die Regung mütterlicher Triebe,\\
Der Vorwitz und der Geist der Liebe\\
Fährt jetzt oft schon in's Flügelkleid.\\
O schlimme Zeit, o schlimme Zeit!

Zu meiner Zeit, zu meiner Zeit\\
ward Pflicht und Ordnung nicht entweiht.\\
Der Mann ward, wie es sich gebühret,\\
Von einer lieben Frau regieret,\\
Trotz seiner stolzen Männlichkeit.\\
O gute Zeit, o gute Zeit!\\
Die Fromme herrschte nur gelinder,\\
Uns blieb der Hut und ihm die Kinder;\\
Das war die Mode weit und breit.\\
O gute Zeit, o gute Zeit!

Zu meiner Zeit, zu meiner Zeit\\
war noch in Ehen Einigkeit.\\
Jetzt darf der Mann uns fast gebieten,\\
Uns widersprechen und uns hüten,\\
Wo man mit Freunden sich erfreut.\\
O schlimme Zeit, o schlimme Zeit!\\
Mit dieser Neuerung im Lande,\\
Mit diesem Fluch im Ehestande\\
Hat ein Komet uns längst bedräut.\\
O schlimme Zeit, o schlimme Zeit!
\end{verse}
\end{otherlanguage}\end{leftcolumn}
\begin{rightcolumn}\begin{otherlanguage}{french}%
\settowidth{\versewidth}{Il y avait encore du droit et de la justice.}%
\begin{verse}[\versewidth]%
\poemtitle{De mon temps.}%
% https://www.lieder.net/lieder/get_text.html?TextId=123236
\noindent
De mon temps, de mon temps,\\
Il y avait encore du droit et de la justice.\\
Là les enfants devenaient des hommes,\\
Et les jeunes filles vertueuses des fiancées ;\\
Mais tout avec modestie.\\
Ô le bon vieux temps, ô le bon vieux temps !\\
Aucun jeune homme ne devenait un traître,\\
Et nos jeunes filles plus tard se mariaient,\\
Elles n'excitaient pas l'envie des mères.\\
Ô le bon vieux temps, ô le bon vieux temps !

De mon temps, de mon temps\\
On faisait attention à être discret.\\
Si le jeune homme éprouvait du plaisir,\\
Il était reconnaissant et gardait le secret ;\\
Mais maintenant il le révèle hardiment ;\\
Ô triste temps, ô triste temps !\\
L'instinct maternel,\\
La curiosité et le désir d'amour\\
Arrivent souvent avant de savoir voler.\\
Ô triste temps, ô triste temps !

De mon temps, de mon temps\\
Le devoir et l'ordre n'étaient pas profanés.\\
Le mari était, comme il se doit,\\
Gouverné par une femme aimée,\\
Malgré sa fierté masculine.\\
Ô le bon vieux temps, ô le bon vieux temps !\\
La femme pieuse régnait plus doucement,\\
À nous la maison, à lui les enfants.\\
C'était une coutume largement répandue.\\
Ô le bon vieux temps, ô le bon vieux temps !

De mon temps, de mon temps\\
Il y avait encore une unité dans le mariage\\
Maintenant le mari peut presque nous commander,\\
Nous contredire et nous garder\\
Quand nous aurions plaisir à être avec des amis.\\
Ô triste temps, ô triste temps !\\
Avec ces innovations dans le pays,\\
Avec cette malédiction sur la vie conjugale,\\
Une comète nous menace pour longtemps.\\
Ô triste temps, ô triste temps !
\end{verse}

\begin{flushright}
\large
\textsc{La Fontaine}, \textit{Fables}.
\end{flushright}
\end{otherlanguage}\end{rightcolumn}
\end{paracol}
\end{document}
